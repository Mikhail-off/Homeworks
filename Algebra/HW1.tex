\documentclass[14pt,a4paper]{scrartcl}
\usepackage[utf8]{inputenc}
\usepackage[english,russian]{babel}
\usepackage{graphicx}
\usepackage{amsmath, amssymb}
\usepackage{setspace}

 \title{
     Домашнее задание по алгебре №1\\
 }
 \author{Михайлов Никита Маратович, ПМИ-167.\\
}
\date{}

\begin{document}
\maketitle
\begin{spacing}{1}

\begin{center}
    \fbox{
        Задание 1.
    }
\end{center}
\noindent \textbf{Решение. } Очевидно, что $m \circ n \in \mathbb{Q}$. Проверим "плохое" равенство: $m \circ n = 1 \Leftrightarrow mn - m - n + 2 = 1 \Leftrightarrow mn - m - n + 1 = 0 \Leftrightarrow m(n - 1) - (n - 1) = 0 \Leftrightarrow (n - 1)(m - 1) = 0 \Leftrightarrow 
\left[\begin{array}{l}
	 n = 1 \\
	 m = 1
\end{array}\right.$, чего быть не может, т.к. $1 \notin \mathbb{Q}\setminus\{1\}$. Следовательно, операция $\circ$ задана на множестве $\mathbb{Q} \setminus \{1\}$. \\
Стоит отметить, что $m \circ n = (m - 1)(n - 1) + 1$.\\
Теперь докажем, что $(\mathbb{Q}\setminus\{1\}, \circ)$ -- группа, проверив соответствующие ей свойства:
\begin{enumerate}
	\item Выделим очень важное свойство: $m \circ n = n \circ m$ -- достаточно очевидно.
	\item $ a \circ (b \circ c) = (a - 1)(b \circ c - 1) + 1 = (a - 1)((b - 1)(c - 1) + 1 - 1) + 1 = (b - 1)(c - 1)(a - 1) + 1$\\
	$(a \circ b) \circ c = ((a - 1)(b - 1) + 1)\circ c =  (((a - 1)(b - 1) + 1) - 1)(c - 1) + 1 = (a - 1)(b - 1)(c - 1) + 1$\\
	Ассоциативность выполнена.
	\item Проверим наличие нейтрального элемента. Обозначим его за $e$, тогда выполняется $\forall a \in \mathbb{Q}\setminus{\{1\}}$: $a \circ e = a \Leftrightarrow ae - a - e + 2 = a \Leftrightarrow  \\ ae - 2a - e + 2 = 0 \Leftrightarrow e(a - 1) = 2(a - 1) \Rightarrow 
	\begin{cases}
	e = \frac12\\
	a \neq 1
	\end{cases} \Rightarrow e = \frac12$
	\item Проверим наличие обратного элемента. Обозначим его $a^{-1}$, тогда выполняется $\forall a \in \mathbb{Q}\setminus\{1\}: a \circ a^{-1} = e \Leftrightarrow aa^{-1} - a - a^{-1} + 2 = e \Leftrightarrow \\ a^{-1}(a - 1) = e + a - 2 \Leftrightarrow
	\begin{cases}
	a^{-1} = \frac{e + a - 2}{a - 2}\\
	a \neq 1
	\end{cases} \Rightarrow a^{-1} = \frac{2a - 3}{2(a - 2)}$
\end{enumerate}
Таким образом, получили, что $(\mathbb{Q} \setminus{\{1\}}, \circ)$ действительно группа.

\end{spacing}

\end{document}