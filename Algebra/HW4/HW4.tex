\documentclass[a4paper, 12pt]{article}
\usepackage[T1]{fontenc}
\usepackage[utf8]{inputenc}
\usepackage{graphicx}
\usepackage{xcolor}


\usepackage[colorlinks=true]{hyperref}
\usepackage{tabularx}

\usepackage{amsmath,amssymb,amsthm,textcomp}
\usepackage{enumerate}
\usepackage{multicol}
\usepackage{tikz}
\usepackage[english, russian]{babel}

\usepackage{cases}

\usepackage{geometry}
\geometry{total={210mm,297mm},
	left=15mm,right=15mm,%
	bindingoffset=0mm, top=20mm,bottom=20mm}

\usepackage{setspace}





\newcommand{\R}{\mathbb{R}} 
\newcommand{\Z}{\mathbb{Z}}
\newcommand{\x}{\times}
\newcommand{\N}{\mathbb{N}} 



\title{
     Домашнее задание по алгебре №3.
 }
 \author{Михайлов Никита Маратович, ПМИ-167.
}
\date{}


\begin{document}
\maketitle
\begin{spacing}{1}


\begin{center}
	\fbox{Задание 1.}
\end{center}

\noindent \textit{Пусть $G$ -- группа всех диагональных матриц в $GL_3(\R)$ и $X = \R^3$. Опишите все орбиты и все стабилизаторы для действия группы $G$ на множестве $X$, заданного формулой $(g, x)\mapsto g \cdot x$.}\\[5pt]
\noindent \textbf{Решение.} Пусть $x \in X$, тогда орбита элемента $x$ -- это $\{gx\;|\;g \in G\}$. Рассмотрим произвольный элемент $g \in G$:
$$
g = \begin{pmatrix}
g_1 & 0 & 0\\
0 & g_2 & 0\\
0 & 0 & g_3
\end{pmatrix}, x = \begin{pmatrix}
x_1 \\ x_2 \\ x_3
\end{pmatrix},
$$
где $g_1, g_2, g_3 \in \R\setminus\{0\}, x_1, x_2, x_3 \in \R $. Посмотрим их произведение:
$$
gx = \begin{pmatrix}
g_1x_1\\ g_2x_2\\ g_3x_3
\end{pmatrix}
$$
Заметим, что если $x_i = 0$, то на i-й координате вектора $x$ будет стоять 0, независимо от $g_i$. Тогда орбиты легко выписать, перебрав все возможные места для нулей:
$$
\begin{pmatrix}
0\\0\\0
\end{pmatrix},
\begin{pmatrix}
a\\0\\0
\end{pmatrix},
\begin{pmatrix}
0\\b\\0
\end{pmatrix},
\begin{pmatrix}
0\\0\\c
\end{pmatrix},
\begin{pmatrix}
a\\b\\0
\end{pmatrix},
\begin{pmatrix}
0\\b\\c
\end{pmatrix},
\begin{pmatrix}
a\\0\\c
\end{pmatrix},
\begin{pmatrix}
a\\b\\c
\end{pmatrix},
$$ где $a,b,c \in \R\setminus\{0\}.$ Теперь опишем стабилизаторы. Пусть $g$ -- стабилизатор точки $x$, тогда:
$$gx=x\Leftrightarrow \begin{pmatrix}
g_1x_1\\g_2x_2\\g_3x_3
\end{pmatrix} = \begin{pmatrix}
x_1\\x_2\\x_3
\end{pmatrix}$$
Рассмотрим уравнение:$g_ix_i=x_i \Rightarrow \begin{cases}
g_i = 1\text{, если } x_i \neq 0\\
g_i \in \R\setminus\{0\}\text{, если }x_i = 0
\end{cases}$ Множество, составленное из удовлетворяющих уравнению $g_i$ будет стабилизатором точки $x$.
\newpage
\begin{center}
	\fbox{Задание 2.}
\end{center}

\noindent \textit{Пусть $G$ -- группа всех верхнетреугольных матриц в $SL_2(\R)$. Опишите все классы сопряженности в группе $G$.}

\noindent \textbf{Решение.} Пусть $g \in G, x \in X = G$, тогда $g = \begin{pmatrix}
g_1 & g_2\\
0 & \frac{1}{g_1}
\end{pmatrix}, x = \begin{pmatrix}
x_1 & x_2\\
0 & \frac{1}{x_1}
\end{pmatrix}, g^{-1} = \begin{pmatrix}
\frac{1}{g_1} &  -g_2\\
0 & g_1
\end{pmatrix}
$. Подействуем сопряжением:
$$
gxg^{-1} = \begin{pmatrix}
g_1x_1 & g_1x_2 + \frac{g_2}{x_1}\\
0 & \frac{1}{g_1x_1}
\end{pmatrix}\cdot \begin{pmatrix}
\frac{1}{g_1} &  -g_2\\
0 & g_1
\end{pmatrix} = \begin{pmatrix}
x_1 & -g_1g_2x_1 + g_1^2x_2 + \frac{g_1g_2}{x_1}\\
0 & \frac{1}{x_1}
\end{pmatrix}
$$
Заметим, что 1) $det(gxg^{-1}) = 1$ (определитель произведения -- произведение определителей);\\ 2) $g_1\neq 0, x_1\neq 0$ -- по условию. Рассмотрим случаи:
\begin{enumerate}
	\item $g, x$ -- недиагональные $\Leftrightarrow \begin{cases}
	g_2 \neq 0\\
	x_2 \neq 0
	\end{cases}$. Ничего интересного с $gxg^{-1}$ не происходит: она может быть как диагональной, так и недиагональной.
	\item $g$ -- диагональная, $x$ -- нет $\Leftrightarrow \begin{cases}
	g_2 = 0\\
	x_2 \neq 0
	\end{cases}$, тогда $gxg^{-1}$ -- недиагональная, так как $g_1^2x_2 \neq 0$
	\item $x$ -- диагональная, $g$ -- нет $\Leftrightarrow \begin{cases}
	g_2 \neq 0\\
	x_2 = 0
	\end{cases}$, тогда рассмотрим выражение:\\
	 $\dfrac{g_1g_2}{x_1} - g_1g_2x_1 = g_1g_2(\dfrac{1}{x_1} - x_1)$. Приравняем его к нулю: $g_1g_2(\dfrac{1}{x_1} - x_1) = 0\Leftrightarrow x_1 = \pm1$. Получили, что $ghg^{-1}$ -- диагональная$\Leftrightarrow x_1 = \pm1$
	 \item $g,x$ -- диагональные $\Leftrightarrow g_2=x_2=0$, тогда и $gxg^{-1}$ -- диагональная.
\end{enumerate}








\newpage
\begin{center}
	\fbox{Задание 3.}
\end{center}

\noindent \textit{Для действия группы $S_4$ на себе сопряжениями найдите стабилизатор подстановки $(1\;2\;3\;4)$.}

\noindent \textbf{Решение.} Пусть $(1\;2\;3\;4) = \sigma = \begin{pmatrix}
1 & 2 & 3 & 4 \\
2 & 3 & 4 & 1
\end{pmatrix}$. Стабилизаторы элемента $\sigma$ -- те элементы $S_4$, к-рые не изменяют $\sigma$ при сопряжении:$\{\delta \in S_4\;|\;\delta \sigma \delta^{-1} = \sigma\}$. Заметим, что $\delta \sigma = \sigma \delta$, а следовательно, $\delta(\sigma(i)) = \sigma(\delta(i))$. Переберем все $i$:
\begin{enumerate}
	\item[$i = 4$:] $\delta(1) = \sigma(\delta(4))$
	\item[$i = 1$:] $\delta(2) = \sigma(\delta(1))$
	\item[$i = 2$:] $\delta(3) = \sigma(\delta(2))$
	\item[$i = 3$:] $\delta(4) = \sigma(\delta(3))$
\end{enumerate}
Теперь переберем все возможные образы для $\delta(4)$:
\begin{enumerate}
	\item $\delta(4) = 1\Rightarrow\begin{cases}
	\delta(1) = \sigma(1) = 2\\
	\delta(2) = \sigma(2) = 3\\
	\delta(3) = \sigma(3) = 4\\
	\delta(4) = \sigma(4) = 1\\
	\end{cases} \Rightarrow \delta = \begin{pmatrix}
	1 & 2 & 3 & 4\\
	2 & 3 & 4 & 1
	\end{pmatrix}$
	\item $\delta(4) = 2\Rightarrow\begin{cases}
	\delta(1) = \sigma(2) = 3\\
	\delta(2) = \sigma(3) = 4\\
	\delta(3) = \sigma(4) = 1\\
	\delta(4) = \sigma(1) = 2\\
	\end{cases} \Rightarrow \delta = \begin{pmatrix}
		1 & 2 & 3 & 4\\
		3 & 4 & 1 & 2
	\end{pmatrix}$
	\item $\delta(4) = 3\Rightarrow\begin{cases}
	\delta(1) = \sigma(3) = 4\\
	\delta(2) = \sigma(4) = 1\\
	\delta(3) = \sigma(1) = 2\\
	\delta(4) = \sigma(2) = 3\\
	\end{cases} \Rightarrow \delta = \begin{pmatrix}
	1 & 2 & 3 & 4\\
	4 & 1 & 2 & 3
	\end{pmatrix}$
	\item $\delta(4) = 4\Rightarrow\begin{cases}
	\delta(1) = \sigma(4) = 1\\
	\delta(2) = \sigma(1) = 2\\
	\delta(3) = \sigma(2) = 3\\
	\delta(4) = \sigma(3) = 4\\
	\end{cases}\Rightarrow \delta = \begin{pmatrix}
	1 & 2 & 3 & 4\\
	1 & 2 & 3 & 4
	\end{pmatrix}$
\end{enumerate}

Все возможные стабилизаторы найдены.






\newpage
\begin{center}
	\fbox{Задание 4.}
\end{center}

\noindent \textit{Пусть $k,l \in \N $ и $n = kl$. Реализуем группу $\Z_k\x\Z_l$ как подгруппу в $S_n$, использую доказательство теоремы Кэли. Найдите необходимое и достаточное условие на числа $k, l$, при котором эта подгруппа содержится в $A_n$.}

\noindent \textbf{Решение.} По теореме Кэли группа $G= \Z_k\x\Z_l \simeq \mathbb{K}\subseteq S_n$. Построим таблицу, отображающая все элементы группы $G$, в которой элемент $(i, j)$ -- находится в строке $i + 1$ и столбце $j + 1$:
$$
\begin{array}{c|cccc}
  & 1 & 2 & \cdots & l\\
  \hline
1 & (0, 0) & (0, 1) & \cdots & (0, l - 1)\\
2 & (1, 0) & (1, 1) & \cdots & (1, l - 1)\\
\vdots & \vdots & \vdots & \ddots &\vdots \\
k & (k - 1, 0) & (k - 1, 1) & \cdots & (k - 1, l - 1)\\
\end{array}
$$
Теперь зададим биекцию между элементами таблицы и натуральными числами (верхней строкой подстановок): $\varphi:G\mapsto [1, n]$ такую, что $\varphi((i, j))\mapsto il + (j + 1)$, где $(i, j)\in G$ . Если интуитивно, то это записать всю таблицу в одну строчку, причем сначала первую строку, потом вторую и т.д.\\ Биекция между $G$ и $K$ определяется следующим образом: 1) есть элемент $(i,j)$; 2) есть группа $G$; 3) ставим элементы группы по возрастанию с помощью биекции $\varphi$; 4) умножаем каждый элемент на $(i, j)$; 5) снова с помощью биекции узнаем номера. Это и будет искомая перестановка.\\
Рассмотрим знак перестановки, полученной из элемента $(0, 1)$. Все строки сдвинутся циклически на 1. Перестановка разобьется на $k$ блоков, в каждом из которых $l- 1$ инверсия. Знак $k(l - 1)$.\\ А теперь для элемента $(1, 0)$: все столбцы сдвинутся на 1. Если восстановить номера, то получится сначала 2я строка, потом 3я, ..., далее 1я. Между строками инверсий нет. Инверсии образуют все первые $k - 1$ блоков с последним. Посчитаем между первым блоком и последним: первый элемент образует с последним $l$ инверсий, второй тоже $l$. Элементов всего в блоке $l$. То есть инверсий $l^2$. Домножим на кол-во блоков. Итого инверсий: $l^2(k - 1)$.\\
Чтобы сделать домножение на $(i, j)$, надо домножить $i$ раз на $(1, 0)$ и $j$ раз на $(0; 1)$. Если они будут четными, то все хорошо и все подстановки тоже будут четными. Составим $\begin{cases}
k(l - 1) \vdots 2\\
l^2(k - 1) \vdots 2
\end{cases} $ Перебрав варианты получили: $l,k$ -- одинаковой четности.


\end{spacing}

\end{document}