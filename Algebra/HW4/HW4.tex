\documentclass[a4paper, 12pt]{article}
\usepackage[T1]{fontenc}
\usepackage[utf8]{inputenc}
\usepackage{graphicx}
\usepackage{xcolor}


\usepackage[colorlinks=true]{hyperref}
\usepackage{tabularx}

\usepackage{amsmath,amssymb,amsthm,textcomp}
\usepackage{enumerate}
\usepackage{multicol}
\usepackage{tikz}
\usepackage[english, russian]{babel}

\usepackage{cases}

\usepackage{geometry}
\geometry{total={210mm,297mm},
	left=15mm,right=15mm,%
	bindingoffset=0mm, top=20mm,bottom=20mm}

\usepackage{setspace}





\newcommand{\R}{\mathbb{R}} 
\newcommand{\Z}{\mathbb{Z}}
\newcommand{\x}{\times}
\newcommand{\N}{\mathbb{N}} 



\title{
     Домашнее задание по алгебре №3.
 }
 \author{Михайлов Никита Маратович, ПМИ-167.
}
\date{}


\begin{document}
\maketitle
\begin{spacing}{1}


\begin{center}
	\fbox{Задание 1.}
\end{center}

\noindent \textit{Пусть $G$ -- группа всех диагональных матриц в $GL_3(\R)$ и $X = \R^3$. Опишите все орбиты и все стабилизаторы для действия группы $G$ на множестве $X$, заданного формулой $(g, x)\mapsto g \cdot x$.}\\[5pt]
\noindent \textbf{Решение.} Пусть $x \in X$, тогда орбита элемента $x$ -- это $\{gx\;|\;g \in G\}$. Рассмотрим произвольный элемент $g \in G$:
$$
g = \begin{pmatrix}
g_1 & 0 & 0\\
0 & g_2 & 0\\
0 & 0 & g_3
\end{pmatrix}, x = \begin{pmatrix}
x_1 \\ x_2 \\ x_3
\end{pmatrix},
$$
где $g_1, g_2, g_3 \in \R\setminus\{0\}, x_1, x_2, x_3 \in \R $. Посмотрим их произведение:
$$
gx = \begin{pmatrix}
g_1x_1\\ g_2x_2\\ g_3x_3
\end{pmatrix}
$$
Заметим, что если $x_i = 0$, то на i-й координате вектора $x$ будет стоять 0, независимо от $g_i$. Тогда орбиты легко выписать, перебрав все возможные места для нулей:
$$
\begin{pmatrix}
0\\0\\0
\end{pmatrix},
\begin{pmatrix}
a\\0\\0
\end{pmatrix},
\begin{pmatrix}
0\\b\\0
\end{pmatrix},
\begin{pmatrix}
0\\0\\c
\end{pmatrix},
\begin{pmatrix}
a\\b\\0
\end{pmatrix},
\begin{pmatrix}
0\\b\\c
\end{pmatrix},
\begin{pmatrix}
a\\0\\c
\end{pmatrix},
\begin{pmatrix}
a\\b\\c
\end{pmatrix},
$$ где $a,b,c \in \R\setminus\{0\}.$ Теперь опишем стабилизаторы. Пусть $g$ -- стабилизатор точки $x$, тогда:
$$gx=x\Leftrightarrow \begin{pmatrix}
g_1x_1\\g_2x_2\\g_3x_3
\end{pmatrix} = \begin{pmatrix}
x_1\\x_2\\x_3
\end{pmatrix}$$
Рассмотрим уравнение:$g_ix_i=x_i \Rightarrow \begin{cases}
g_i = 1\text{, если } x_i \neq 0\\
g_i \in \R\setminus\{0\}\text{, если }x_i = 0
\end{cases}$ Множество из $g$, составленных из удовлетворяющих уравнению $g_i$ будет стабилизатором точки $x$.
\newpage
\begin{center}
	\fbox{Задание 2.}
\end{center}

\noindent \textit{Пусть $G$ -- группа всех верхнетреугольных матриц в $SL_2(\R)$. Опишите все классы сопряженности в группе $G$.}

\noindent \textbf{Решение.}









\newpage
\begin{center}
	\fbox{Задание 3.}
\end{center}

\noindent \textit{Для действия группы $S_4$ на себе сопряжениями найдите стабилизатор подстановки $(1\;2\;3\;4)$.}

\noindent \textbf{Решение.} Пусть $(1\;2\;3\;4) = \sigma = \begin{pmatrix}
1 & 2 & 3 & 4 \\
2 & 3 & 4 & 1
\end{pmatrix}$. Стабилизаторы элемента $\sigma$ -- те элементы $S_4$, к-рые не изменяют $\sigma$ при сопряжении:$\{\delta \in S_4\;|\;\delta \sigma \delta^{-1} = \sigma\}$. Заметим, что $\delta \sigma = \sigma \delta$, а следовательно, $\delta(\sigma(i)) = \sigma(\delta(i))$. Переберем все $i$:
\begin{enumerate}
	\item[$i = 4$:] $\delta(1) = \sigma(\delta(4))$
	\item[$i = 1$:] $\delta(2) = \sigma(\delta(1))$
	\item[$i = 2$:] $\delta(3) = \sigma(\delta(2))$
	\item[$i = 3$:] $\delta(4) = \sigma(\delta(3))$
\end{enumerate}
Теперь переберем все возможные образы для $\delta(4)$:
\begin{enumerate}
	\item $\delta(4) = 1\Rightarrow\begin{cases}
	\delta(1) = \sigma(1) = 2\\
	\delta(2) = \sigma(2) = 3\\
	\delta(3) = \sigma(3) = 4\\
	\delta(4) = \sigma(4) = 1\\
	\end{cases} \Rightarrow \delta = \begin{pmatrix}
	1 & 2 & 3 & 4\\
	2 & 3 & 4 & 1
	\end{pmatrix}$
	\item $\delta(4) = 2\Rightarrow\begin{cases}
	\delta(1) = \sigma(2) = 3\\
	\delta(2) = \sigma(3) = 4\\
	\delta(3) = \sigma(4) = 1\\
	\delta(4) = \sigma(1) = 2\\
	\end{cases} \Rightarrow \delta = \begin{pmatrix}
		1 & 2 & 3 & 4\\
		3 & 4 & 1 & 2
	\end{pmatrix}$
	\item $\delta(4) = 3\Rightarrow\begin{cases}
	\delta(1) = \sigma(3) = 4\\
	\delta(2) = \sigma(4) = 1\\
	\delta(3) = \sigma(1) = 2\\
	\delta(4) = \sigma(2) = 3\\
	\end{cases} \Rightarrow \delta = \begin{pmatrix}
	1 & 2 & 3 & 4\\
	4 & 1 & 2 & 3
	\end{pmatrix}$
	\item $\delta(4) = 4\Rightarrow\begin{cases}
	\delta(1) = \sigma(4) = 1\\
	\delta(2) = \sigma(1) = 2\\
	\delta(3) = \sigma(2) = 3\\
	\delta(4) = \sigma(3) = 4\\
	\end{cases}\Rightarrow \delta = \begin{pmatrix}
	1 & 2 & 3 & 4\\
	1 & 2 & 3 & 4
	\end{pmatrix}$
\end{enumerate}

Все возможные стабилизаторы найдены.






\newpage
\begin{center}
	\fbox{Задание 4.}
\end{center}

\noindent \textit{Пусть $k,l \in \N $ и $n = kl$. Реализуем группу $\Z_k\x\Z_l$ как подгруппу в $SL_n$, использую доказательство теоремы Кэли. Найдите необходимое и достаточное условие на числа $k, l$, при котором эта подгруппа содержится в $A_n$.}

\noindent \textbf{Решение.} 




\end{spacing}
\end{document}