\documentclass[a4paper, 12pt]{article}
\usepackage[T1]{fontenc}
\usepackage[utf8]{inputenc}
\usepackage{graphicx}
\usepackage{xcolor}


\usepackage[colorlinks=true]{hyperref}
\usepackage{tabularx}

\usepackage{amsmath,amssymb,amsthm,textcomp}
\usepackage{enumerate}
\usepackage{multicol}
\usepackage{tikz}
\usepackage[english, russian]{babel}

\usepackage{cases}

\usepackage{geometry}
\geometry{total={210mm,297mm},
	left=15mm,right=15mm,%
	bindingoffset=0mm, top=20mm,bottom=20mm}

\usepackage{setspace}




\newcommand{\N}{\mathbb{N}} 
\newcommand{\Z}{\mathbb{Z}}
\newcommand{\x}{\times}




\title{
     Домашнее задание по алгебре №3.
 }
 \author{Михайлов Никита Маратович, ПМИ-167.
}
\date{}


\begin{document}
\maketitle
\begin{spacing}{1}


\begin{center}
	\fbox{Задание 1.}
\end{center}

\noindent Сколько элементов	порядков 2, 3, 4 и 6 в группе $\mathbb{Z}_3 \times \mathbb{Z}_4\times \mathbb{Z}_6$?\\

\noindent \textbf{Решение. } Заметим, что элемент группы $\mathbb{Z}_3 \times \mathbb{Z}_4\times \mathbb{Z}_6$ имеет порядок $n$ тогда и только тогда, когда $\text{НОК}(ord(a), ord(b), ord(c)) = n$, где $a\in\mathbb{Z}_3,b\in\mathbb{Z}_4,c\in\mathbb{Z}_6.$
\begin{enumerate}
	\item[Порядка 2.] Перечислим все элементы групп, которые в степени 2 дают 0:
	$$
	\begin{array}{c|c|c}
		\mathbb{Z}_3 & \mathbb{Z}_4 & \mathbb{Z}_6\\
		\hline 
		0 & 0, 2 & 0, 3\\
	\end{array}
	$$
	Теперь на первое место можем поставить 1 элемент, на второе 2, на третье 2. И вычтем 1, так как элемент (0, 0, 0) -- единственный(пока) случай, когда НОК < 2.\\
	Итого: $1\cdot2\cdot2 - 1 = 3$
	
	\item[Порядка 3.] Аналогично перечислим:
	$$
	\begin{array}{c|c|c}
	\mathbb{Z}_3 & \mathbb{Z}_4 & \mathbb{Z}_6\\
	\hline 
	0,1,2 & 0 & 0, 2, 4\\
	\end{array}
	$$
	Итого: $3\cdot 1\cdot3 - 1 = 8$
	\item[Порядка 4.] Снова перечислим:
	$$
	\begin{array}{c|c|c}
	\mathbb{Z}_3 & \mathbb{Z}_4 & \mathbb{Z}_6\\
	\hline 
	0 & 0,1,2,3 & 0, 3\\
	\end{array}
	$$
	Итого: $1\cdot4\cdot2 - 1 = 7$. Но еще нужно вычесть все элементы порядка, делящих 4, а именно порядка 2, ведь такие элементы в степени 4 тоже дают 0. Получили: $7 - 3 = 4$.
	
	\item[Порядка 6.] Перечислим:
	$$
	\begin{array}{c|c|c}
	\mathbb{Z}_3 & \mathbb{Z}_4 & \mathbb{Z}_6\\
	\hline 
	0,1,2 & 0,2 & 0,1,2,3,4,5\\
	\end{array}
	$$
	Итого: $3\cdot2\cdot6 - 1 - 8 -3 = 24$.
\end{enumerate}


\begin{center}
	\fbox{Задание 2.}
\end{center}

\noindent Сколько подгрупп порядков 3 и 15 в нециклической абелевой группе порядка 45?\\

\noindent \textbf{Решение.} \textit{Основополагающая теорема о структуре конечной абелевой группы утверждает, что любая конечная абелева группа может быть разложена в прямую сумму своих циклических подгрупп, порядки которых являются степенями простых чисел.}\\
\noindent Следовательно, $G$ изоморфна одной из $\mathbb{Z}_{45},\mathbb{Z}_3 \times \mathbb{Z}_3 \times \mathbb{Z}_5, \mathbb{Z}_9 \times \mathbb{Z}_5, \mathbb{Z}_3 \times \mathbb{Z}_{15}$.\\
Заметим, что 1) $\mathbb{Z}_{45}\simeq\mathbb{Z}_9 \times\mathbb{Z}_5$;
2) $\mathbb{Z}_3 \times \mathbb{Z}_3 \times \mathbb{Z}_5 \simeq \mathbb{Z}_3 \times \mathbb{Z}_{15}$, причем $\mathbb{Z}_3 \times \mathbb{Z}_{15}$ -- ациклическая, так как $\text{НОД}(3, 15) \neq 1$\\
Таким образом $G \simeq \mathbb{Z}_3 \times \mathbb{Z}_3 \times \mathbb{Z}_5$.
\begin{enumerate}
	\item[Порядка 3.] Аналогично №1 найдем кол-во элементов порядка 3. $3\times3\times1 - 1 = 8$. Но так же отметим, что подгруппа порядка 3 порождается любым своим неединичным элементом, так как 3 -- простое(следствие т. Лагранжа). Тогда получается, что мы посчитали все подгруппы дважды. Итого: $\frac{8}{2} = 4$.
	
	\item[Порядка 15.] Любая абелева группа порядка 15 изоморфна либо $\mathbb{Z}_{15}$, либо $\mathbb{Z}_3 \times \mathbb{Z}_5$, но они изоморфны. Следовательно, каждая подгруппа в точности изоморфна $\mathbb{Z}_{15}$. А у $\mathbb{Z}_{15}$ ровно $\varphi(15) = 8$ порождающих элементов. Теперь найдем кол-во элементов порядка 15 исходной группы аналогично №1: $3\cdot3\cdot5 - 8 - 4 - 1 = 32$. И разделим на 4, так как мы посчитаем каждую подгруппу четырежды. Итого:$\frac{32}{8} = 4$
\end{enumerate}


\begin{center}
	\fbox{Задание 3.}
\end{center}

\noindent Найдите в группе $G = \mathbb{Z}\times \mathbb{Z}$ подгруппу $H$, для которой $G/H\simeq\mathbb{Z}_{10}\times\mathbb{Z}_{12}\times\mathbb{Z}_{15}$.\\

\noindent \textbf{Решение.} Заметим, что $\mathbb{Z}_{10}\times\mathbb{Z}_{12}\times\mathbb{Z}_{15} \simeq \Z_2 \x \Z_5 \x \Z_3 \x \Z_4 \x \Z_3 \x \Z_5 \simeq \Z_{30}\x\Z_{60}$.\\
Известно, что $\begin{cases}
	G' = \Z\\
	H' = n\Z
\end{cases}\Rightarrow G'/H' \simeq \Z_n$. Пусть $H = H_1 \x H_2$. Тогда возьмем $\begin{cases}
	H_1 = \Z_{30}\\
	H_2 = \Z_{60}
\end{cases}$, причем $H_1$ и $H_2$ нормальны в $\Z$, тогда $H = \Z_{30} \x \Z_{60}$. По т. о факторизации:
$$
G/H \simeq \Z / H_1 \x \Z/H_2 \simeq \Z_{30}\x \Z_{60} \simeq \mathbb{Z}_{10}\times\mathbb{Z}_{12}\times\mathbb{Z}_{15}
$$




\begin{center}
	\fbox{Задание 4.}
\end{center}

\noindent Пусть порядок конечной абелевой группы $A$ делится на $m$. Докажите, что в $A$ есть подгруппа порядка $m$.\\

\noindent \textbf{Решение.} Для $m = 1$ все очевидно. Пусть $m \neq 1$.\\
Применим метод математической индукции. База: если $G$ -- циклическая абелева группа порядка $N$, порожденная элементом $g$, тогда $\forall n: N \vdots n \Rightarrow ord(g^{\frac{N}{n}}) = n$. Пусть верно для всех порядков меньших $N$. Если $G$ -- циклическая, то все хорошо. $G$ -- ациклическая. Следовательно, по структурной теореме она изоморфна прямому произведению некоторых абелевых групп \underline{меньших} порядков:
$G = G_1\x G_2$, где $\begin{cases}
	|G_1| = x \\
	|G_2| = y
\end{cases}$. Имеем: $\begin{cases}
    N = xy\\
    N \vdots m
\end{cases}$. Из курса дискретной математики $\exists x', y': m = x'y'$, причем $x\vdots x'$ и $y \vdots y'$, тогда существуют такие $H_1\subseteq G_1, H_2\subseteq G_2$, что $|H_1| = x'$ и $|H_2| = y'$, тогда $H_1\x H_2 \subseteq G$, а $|H_1\x H_2| = x'y' = m$. Что и требовалось доказать.


\end{spacing}

\end{document}