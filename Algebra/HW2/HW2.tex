\documentclass[14pt,a4paper]{scrartcl}
\usepackage[utf8]{inputenc}
\usepackage[english,russian]{babel}
\usepackage{graphicx}
\usepackage{amsmath, amssymb}
\usepackage{setspace}

\title{
 Домашнее задание по алгебре №2\\
}
\author{
	Михайлов Никита Маратович, ПМИ-167.\\
}
\date{}

\begin{document}
\maketitle
\begin{spacing}{1}
	\begin{center}
		\fbox{
			Задание 1.
		}
	\end{center}
	\textit{Найдите все левые смежные классы и все правые смежные классы группы
		$A_4$ по подгруппе $H = \langle \sigma \rangle$, где $\sigma =
		\left(
		\begin{array}{cccc}
			1 & 2 & 3 & 4 \\
			2& 1 & 4 & 3 
		\end{array}
		\right)$. Является ли подгруппа $H$ нормальной в группе $A_4$?} \\
	\noindent \underline{\textbf{Решение.}} $\sigma = \tau_{12}\cdot\tau_{34} \in A_4$. Заметим, что $\langle \sigma \rangle = \{id, \sigma\}$, т.к. 
	$\begin{cases}
		\sigma^{2k} = id \\
		\sigma^{2k+1} = \sigma
	\end{cases}$. 
	Для элемента $id$ левый смежный класс группы $A_4$ по $id$ совпадает с правым и совпадает с $A_4$. Осталось рассмотреть левый и правый смежные классы группы $A_4$ по $\sigma$. Для начала найдем все элементы группы $A_4$(их $\frac{4!}{2} = 12$):
	\begin{gather*}
	\begin{pmatrix}
	1 & 2 & 3 & 4 \\
	1 & 2 & 3 & 4
	\end{pmatrix},
	\begin{pmatrix}
	1 & 2 & 3 & 4 \\
	2 & 1 & 4 & 3
	\end{pmatrix},
	\begin{pmatrix}
	1 & 2 & 3 & 4 \\
	3 & 4 & 1 & 2
	\end{pmatrix}
	\begin{pmatrix}
	1 & 2 & 3 & 4 \\
	4 & 3 & 2 & 1
	\end{pmatrix}, \\
	\begin{pmatrix}
	1 & 2 & 3 & 4 \\
	4 & 2 & 1 & 3
	\end{pmatrix},
	\begin{pmatrix}
	1 & 2 & 3 & 4 \\
	3 & 1 & 2 & 4
	\end{pmatrix},
	\begin{pmatrix}
	1 & 2 & 3 & 4 \\
	2 & 3 & 1 & 4
	\end{pmatrix},
	\begin{pmatrix}
	1 & 2 & 3 & 4 \\
	4 & 1 & 3 & 2
	\end{pmatrix}, \\
	\begin{pmatrix}
	1 & 2 & 3 & 4 \\
	1 & 3 & 4 & 2
	\end{pmatrix},
	\begin{pmatrix}
	1 & 2 & 3 & 4 \\
	2 & 4 & 3 & 1
	\end{pmatrix},
	\begin{pmatrix}
	1 & 2 & 3 & 4 \\
	3 & 2 & 4 & 1
	\end{pmatrix},
	\begin{pmatrix}
	1 & 2 & 3 & 4 \\
	1 & 4 & 2 & 3
	\end{pmatrix},	
	\end{gather*}
	
\end{spacing}

\end{document}
