\documentclass[14pt,a4paper]{scrartcl}
\usepackage[utf8]{inputenc}
\usepackage[english,russian]{babel}
\usepackage{graphicx}
\usepackage{amsmath, amssymb}
\usepackage{setspace}

\title{
 Домашнее задание по алгебре №2\\
}
\author{
	Михайлов Никита Маратович, ПМИ-167.\\
}
\date{}

\begin{document}
\maketitle
\begin{spacing}{1}
	\begin{center}
		\fbox{
			Задание 1.
		}
	\end{center}
	\textit{Найдите все левые смежные классы и все правые смежные классы группы
		$A_4$ по подгруппе $H = \langle \sigma \rangle$, где $\sigma =
		\left(
		\begin{array}{cccc}
			1 & 2 & 3 & 4 \\
			2& 1 & 4 & 3 
		\end{array}
		\right)$. Является ли подгруппа $H$ нормальной в группе $A_4$?} \\
	\noindent \underline{\textbf{Решение.}} $\sigma = \tau_{12}\cdot\tau_{34} \in A_4$. Заметим, что $\langle \sigma \rangle = \{id, \sigma\}$, т.к. 
	$\begin{cases}
		\sigma^{2k} = id \\
		\sigma^{2k+1} = \sigma
	\end{cases}$. 
	Для начала найдем все элементы группы $A_4$(их $\frac{4!}{2} = 12$):
	\begin{gather*}
	a_1 = \begin{pmatrix}
	1 & 2 & 3 & 4 \\
	1 & 2 & 3 & 4
	\end{pmatrix},
	a_2 = \begin{pmatrix}
	1 & 2 & 3 & 4 \\
	2 & 1 & 4 & 3
	\end{pmatrix},
	a_3 = \begin{pmatrix}
	1 & 2 & 3 & 4 \\
	3 & 4 & 1 & 2
	\end{pmatrix}
	a_4 = \begin{pmatrix}
	1 & 2 & 3 & 4 \\
	4 & 3 & 2 & 1
	\end{pmatrix}, \\
	a_5 = \begin{pmatrix}
	1 & 2 & 3 & 4 \\
	4 & 2 & 1 & 3
	\end{pmatrix},
	a_6 = \begin{pmatrix}
	1 & 2 & 3 & 4 \\
	3 & 1 & 2 & 4
	\end{pmatrix},
	a_7 = \begin{pmatrix}
	1 & 2 & 3 & 4 \\
	2 & 3 & 1 & 4
	\end{pmatrix},
	a_8 = \begin{pmatrix}
	1 & 2 & 3 & 4 \\
	4 & 1 & 3 & 2
	\end{pmatrix}, \\
	a_9 = \begin{pmatrix}
	1 & 2 & 3 & 4 \\
	1 & 3 & 4 & 2
	\end{pmatrix},
	a_{10} = \begin{pmatrix}
	1 & 2 & 3 & 4 \\
	2 & 4 & 3 & 1
	\end{pmatrix},
	a_{11} = \begin{pmatrix}
	1 & 2 & 3 & 4 \\
	3 & 2 & 4 & 1
	\end{pmatrix},
	a_{12} = \begin{pmatrix}
	1 & 2 & 3 & 4 \\
	1 & 4 & 2 & 3
	\end{pmatrix},	
	\end{gather*}
	$|H| = 2 \Rightarrow |L_i| = |R_i| = 2$ для всех $i = \{1, ..., 12\}$, где $L_i$ -- левый, а $R_i$ -- правый смежные классы i-ого элемента $A_4$.
	\begin{enumerate}
		\item $L_1 = \left\{ 
		\begin{pmatrix}
		1 & 2 & 3 & 4 \\
		1 & 2 & 3 & 4
		\end{pmatrix},
		\begin{pmatrix}
		1 & 2 & 3 & 4 \\
		2 & 1 & 4 & 3
		\end{pmatrix}
		\right\};
		R_1 = \left\{ 
		\begin{pmatrix}
		1 & 2 & 3 & 4 \\
		1 & 2 & 3 & 4
		\end{pmatrix},
		\begin{pmatrix}
		1 & 2 & 3 & 4 \\
		2 & 1 & 4 & 3
		\end{pmatrix}
		\right\}$
		
		\item $L_2 = \left\{ 
		\begin{pmatrix}
			1 & 2 & 3 & 4 \\
			1 & 2 & 3 & 4
		\end{pmatrix},
		\begin{pmatrix}
			1 & 2 & 3 & 4 \\
			2 & 1 & 4 & 3
		\end{pmatrix}
		\right\};
		R_2 = \left\{ 
		\begin{pmatrix}
			1 & 2 & 3 & 4 \\
			1 & 2 & 3 & 4
		\end{pmatrix},
		\begin{pmatrix}
			1 & 2 & 3 & 4 \\
			2 & 1 & 4 & 3
		\end{pmatrix}
		\right\};$
		
		\item $L_3 = \left\{ 
		\begin{pmatrix}
		1 & 2 & 3 & 4 \\
		3 & 4 & 1 & 2
		\end{pmatrix},
		\begin{pmatrix}
		1 & 2 & 3 & 4 \\
		4 & 3 & 2 & 1
		\end{pmatrix}
		\right\};
		R_3 = \left\{ 
		\begin{pmatrix}
		1 & 2 & 3 & 4 \\
		3 & 4 & 1 & 2
		\end{pmatrix},
		\begin{pmatrix}
		1 & 2 & 3 & 4 \\
		4 & 3 & 2 & 1
		\end{pmatrix}
		\right\};$
		
		\item $L_4 = \left\{ 
		\begin{pmatrix}
		1 & 2 & 3 & 4 \\
		4 & 3 & 2 & 1
		\end{pmatrix},
		\begin{pmatrix}
		1 & 2 & 3 & 4 \\
		3 & 4 & 1 & 2
		\end{pmatrix}
		\right\};
		R_4 = \left\{ 
		\begin{pmatrix}
		1 & 2 & 3 & 4 \\
		4 & 3 & 2 & 1
		\end{pmatrix},
		\begin{pmatrix}
		1 & 2 & 3 & 4 \\
		3 & 4 & 1 & 2
		\end{pmatrix}
		\right\};$
		
		\item $L_5 = \left\{ 
		\begin{pmatrix}
		1 & 2 & 3 & 4 \\
		4 & 2 & 1 & 3
		\end{pmatrix},
		\begin{pmatrix}
		1 & 2 & 3 & 4 \\
		2 & 4 & 3 & 1
		\end{pmatrix}
		\right\};
		R_5 = \left\{ 
		\begin{pmatrix}
		1 & 2 & 3 & 4 \\
		4 & 2 & 1 & 3
		\end{pmatrix},
		\begin{pmatrix}
		1 & 2 & 3 & 4 \\
		3 & 1 & 2 & 4
		\end{pmatrix}
		\right\};$
		
		\item $L_6 = \left\{ 
		\begin{pmatrix}
		1 & 2 & 3 & 4 \\
		3 & 1 & 2 & 4
		\end{pmatrix},
		\begin{pmatrix}
		1 & 2 & 3 & 4 \\
		1 & 3 & 4 & 2
		\end{pmatrix}
		\right\};
		R_6 = \left\{ 
		\begin{pmatrix}
		1 & 2 & 3 & 4 \\
		3 & 1 & 2 & 4
		\end{pmatrix},
		\begin{pmatrix}
		1 & 2 & 3 & 4 \\
		4 & 2 & 1 & 3
		\end{pmatrix}
		\right\};$
		
		\item $L_7 = \left\{ 
		\begin{pmatrix}
		1 & 2 & 3 & 4 \\
		2 & 3 & 1 & 4
		\end{pmatrix},
		\begin{pmatrix}
		1 & 2 & 3 & 4 \\
		3 & 2 & 4 & 1
		\end{pmatrix}
		\right\};
		R_7 = \left\{ 
		\begin{pmatrix}
		1 & 2 & 3 & 4 \\
		2 & 3 & 1 & 4
		\end{pmatrix},
		\begin{pmatrix}
		1 & 2 & 3 & 4 \\
		1 & 4 & 2 & 3
		\end{pmatrix}
		\right\};$
		
		\item $L_8 = \left\{ 
		\begin{pmatrix}
		1 & 2 & 3 & 4 \\
		4 & 1 & 3 & 2
		\end{pmatrix},
		\begin{pmatrix}
		1 & 2 & 3 & 4 \\
		1 & 4 & 2 & 3
		\end{pmatrix}
		\right\};
		R_8 = \left\{ 
		\begin{pmatrix}
		1 & 2 & 3 & 4 \\
		4 & 1 & 3 & 2
		\end{pmatrix},
		\begin{pmatrix}
		1 & 2 & 3 & 4 \\
		1 & 2 & 3 & 4
		\end{pmatrix}
		\right\};$
		
		\item $L_9 = \left\{ 
		\begin{pmatrix}
		1 & 2 & 3 & 4 \\
		1 & 3 & 4 & 2
		\end{pmatrix},
		\begin{pmatrix}
		1 & 2 & 3 & 4 \\
		3 & 1 & 2 & 4
		\end{pmatrix}
		\right\};
		R_9 = \left\{ 
		\begin{pmatrix}
		1 & 2 & 3 & 4 \\
		1 & 3 & 4 & 2
		\end{pmatrix},
		\begin{pmatrix}
		1 & 2 & 3 & 4 \\
		2 & 4 & 3 & 1
		\end{pmatrix}
		\right\};$
		
		\item $L_{10} = \left\{ 
		\begin{pmatrix}
		1 & 2 & 3 & 4 \\
		2 & 4 & 3 & 1
		\end{pmatrix},
		\begin{pmatrix}
		1 & 2 & 3 & 4 \\
		4 & 2 & 1 & 3
		\end{pmatrix}
		\right\};
		R_{10} = \left\{ 
		\begin{pmatrix}
		1 & 2 & 3 & 4 \\
		2 & 4 & 3 & 1
		\end{pmatrix},
		\begin{pmatrix}
		1 & 2 & 3 & 4 \\
		1 & 3 & 4 & 2
		\end{pmatrix}
		\right\};$
		
		\item $L_{11} = \left\{ 
		\begin{pmatrix}
		1 & 2 & 3 & 4 \\
		3 & 2 & 4 & 1
		\end{pmatrix},
		\begin{pmatrix}
		1 & 2 & 3 & 4 \\
		2 & 3 & 1 & 4
		\end{pmatrix}
		\right\};
		R_{11} = \left\{ 
		\begin{pmatrix}
		1 & 2 & 3 & 4 \\
		3 & 2 & 4 & 1
		\end{pmatrix},
		\begin{pmatrix}
		1 & 2 & 3 & 4 \\
		4 & 1 & 3 & 2
		\end{pmatrix}
		\right\};$
		
		\item $L_{12} = \left\{ 
		\begin{pmatrix}
		1 & 2 & 3 & 4 \\
		1 & 4 & 2 & 3
		\end{pmatrix},
		\begin{pmatrix}
		1 & 2 & 3 & 4 \\
		4 & 1 & 3 & 2
		\end{pmatrix}
		\right\};
		R_{12} = \left\{ 
		\begin{pmatrix}
		1 & 2 & 3 & 4 \\
		1 & 4 & 2 & 3
		\end{pmatrix},
		\begin{pmatrix}
		1 & 2 & 3 & 4 \\
		2 & 3 & 1 & 4
		\end{pmatrix}
		\right\};$
	\end{enumerate}
	Заметим, что существуют такие $a_i$ для которых $a_iH \neq Ha_i \Rightarrow$ подгруппа $H$ -- ненормальна.
	
	
\begin{center}
	\fbox{
		Задание 2.
	}
\end{center}
	\textit{Пусть $SL_2$ -- группа всех целочисленных $(2\times 2)$-матриц с определителем 1. Докажите, что множество
		$$
		\left\{\begin{pmatrix}
		a & b\\
		c & d
		\end{pmatrix} \in SL_2(\mathbb{Z})\;|\;a \equiv d \equiv 1\; (mod\; 3);\;	b \equiv c \equiv 0\;(mod\;3)\right\}
		$$}
	является нормальной группой в $SL_2(\mathbb{Z})$.\\
	\noindent \textbf{\underline{Решение.}} Обозначим нашу группу за $G$. Рассмотрим произвольную матрицу из $G :g =  \begin{pmatrix}
	x_1 & x_2\\
	x_3 & x_4
	\end{pmatrix}$. Так как $G$ -- группа, то там лежит и обратная матрица: $g^{-1} = \begin{pmatrix}
	x_4 & -x_2\\
	-x_3 & x_1
	\end{pmatrix}$. По определению нашей группы определители обоих матриц равны 1: $x_1x_4 - x_2x_3 = 1$. Покажем, что $gHg^{-1} \subseteq H$, где $H$ -- подгруппа из условия.(очевидно, что это группа, так как обратная матрица любого элемента $H$ на диагонали содержир те же элементы, а на побочной минус на сравнимость с нулем не влияет. Единичная матрица так же удовлетворяет всем свойствам). Рассмотрим произвольный элемент из $H$: $h = \begin{pmatrix}
		a & b \\
		c & d
	\end{pmatrix} $
	Таким образом:
	$$
	ghg^{-1} = \begin{pmatrix}
	x_1 & x_2\\
	x_3 & x_4
	\end{pmatrix} \cdot 
	\begin{pmatrix}
	a & b \\
	c & d
	\end{pmatrix} \cdot
	\begin{pmatrix}
	x_4 & -x_2\\
	-x_3 & x_1
	\end{pmatrix} = 
	\begin{pmatrix}
	x_1a+x_2c & x_1b+x_2d\\
	x_3a + x_4c & x_3b + x_4d
	\end{pmatrix}	\cdot
	\begin{pmatrix}
	x_4 & -x_2\\
	-x_3 & x_1
	\end{pmatrix}
	$$
	Используя определения нашей группы и ее подгруппы:
	\begin{enumerate}
		\item $(x_1a+x_2c)x_4 - (x_1b+x_2d)x_3 = x_1x_4a + x_2x_4c - x_1x_3b-x_2x_3d \equiv x_1x_4a-x_2x_3d \equiv (1 + x_2x_3)a - x_2x_3d \equiv a + x_2x_3(a - d) \equiv a \equiv 1\;(mod\;3)$
		\item $-(x_1a+x_2c)x_2 + (x_1b+x_2d)x_1 = -x_1x_2a - x_2x_2c + x_1x_1b+x_2x_1d \equiv x_1^2b - x_2^2c \equiv 0\;(mod\;3)$
		\item Нижняя строка получившейся матрицы тоже удовлетворяет свойствам выше. Проверяется аналогично двум случаям выше заменой соответствующих индексов.
	\end{enumerate}
	Получили, что подгруппа $H$ нормальна по одной из формулировок нормальности(их несколько).
	
	
	\begin{center}
		\fbox{
			Задание 3.
		}
	\end{center}
	\textit{Найдите все гомоморфизмы из группы $\mathbb{Z}_{12}$ в группу $\mathbb{Z}_{16}$.}
	
	\noindent \textbf{\underline{Решение.}} Заметим, что если мы будем знать $\varphi(1)$, то мы сможем вычислить $\varphi(x) = x\varphi(1)$. Довольно очевидно, что 0 перейдет в 0. Но в $\mathbb{Z}_{12}\;12 = 0 \Rightarrow \varphi(0) = \varphi(12) = 12\varphi(1) = 0\;(mod\;16)$. Различных $\varphi(1)$ по модулю 16 немного, а именно: $\{0; 4; 8; 12\}$. В силу первого предложения можно сказать, что мы нашли все гомоморфизмы. 
	
	\begin{center}
		\fbox{
			Задание 4.
		}
	\end{center}
	\textit{Перечислите все с точностью до изоморфизма группы, каждая из которых изоморфна любой своей неедининой подгруппе.}\\
	\noindent \textbf{\underline{Решение.}} Заметим, что группа -- не конечна(разные мощности -- нет биекции). Пусть наша группа $G$. Тогда рассмотрим ее подгруппу, порожденную элементом $g$: $H = \langle g \rangle$. $H$ циклична и бесконечна. Но по условию $G \simeq H$. Следовательно, $G$ тоже циклична и бесконечна. Как мы знаем все бесконечные циклические группы изоморфны $\mathbb{Z}$($k\mathbb{Z}$).
\end{spacing}

\end{document}
