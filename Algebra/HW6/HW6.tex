\documentclass[a4paper, 12pt]{article}
\usepackage[T1]{fontenc}
\usepackage[utf8]{inputenc}
\usepackage{graphicx}
\usepackage{xcolor}


\usepackage[colorlinks=true]{hyperref}
\usepackage{tabularx}

\usepackage{amsmath,amssymb,amsthm,textcomp}
\usepackage{enumerate}
\usepackage{multicol}
\usepackage{tikz}
\usepackage[english, russian]{babel}

\usepackage{cases}

\usepackage{geometry}
\geometry{total={210mm,297mm},
	left=15mm,right=15mm,%
	bindingoffset=0mm, top=20mm,bottom=20mm}

\usepackage{setspace}





\newcommand{\R}{\mathbb{R}} 
\newcommand{\Z}{\mathbb{Z}}
\newcommand{\x}{\times}
\newcommand{\N}{\mathbb{N}} 
\newcommand{\Co}{\mathbb{C}}



\title{
     Домашнее задание по алгебре №6.
 }
 \author{Михайлов Никита Маратович, ПМИ-167.
}
\date{}


\begin{document}
\maketitle
\begin{spacing}{1}


\begin{center}
	\fbox{Задание 1.}
\end{center}

\noindent \textit{Найдите наибольший общий делитель многочленов
$$
f(x) = 2x^4-4x^3-3x^2+7x-2 \text{ и } g(x) = 6x^3+4x^2-5x+1,
$$
а так же его линейное выражение через $f(x)$ и $g(x)$.
}\\
\noindent \textbf{Решение.} Применим расширенный алгоритм Евклида (опустим школьные вычисления столбиком).
\begin{enumerate}
	\item $\displaystyle f(x) = g(x) (\frac{1}{3}x - \frac{8}{9}) + (\frac{20}{9}x^2+\frac{20}{9}x - \frac{10}{9})$
	\item $\displaystyle g(x) = (\frac{20}{9}x^2+\frac{20}{9}x - \frac{10}{9})(\frac{27}{10}x-\frac{9}{10})+0$
\end{enumerate}
Следовательно, НОД$\big(f(x), g(x)\big) = \dfrac{20}{9}x^2+\dfrac{20}{9}x - \dfrac{10}{9}$. А его линейное выражение легко выразить из шага №1:
$$
\text{НОД}\big(f(x), g(x)\big) = 1\cdot f(x) + (-\dfrac{1}{3}x + \dfrac{8}{9})\cdot g(x)
$$


\begin{center}
	\fbox{Задание 2.}
\end{center}

\noindent \textit{Разложите многочлен $x^6+x^3-12$ в произведение неприводимых в кольце $\Co [x]$ и в кольце $\R[x]$.}\\

\noindent \textbf{Решение.} Известно, что в кольце $\Co[x]$ неприводимыми являются многочлены степени 1. По основной теореме алгебры данный многочлен $f(x) = x^6 + x^3 - 12$ раскладывается в произведение многочленов типа: $x-x_0$, где $x_0$ -- корень уравнения $f(x) = 0$. Решим это уравнение. \\
Составим $x^6 + x^3 - 12 = 0$. Обозначим $x^3 = t$, тогда $t^2 + t - 12 = 0 \Rightarrow t_{1,2} = \dfrac{-1 \pm \sqrt{1 + 48}}{2} = \dfrac{-1 \pm 7}{2} \Rightarrow \left[\begin{array}{l|}
	t_1 = -4\\
	t_2 = 3
\end{array}\right.
$ Перейдем к $x$: $\left[\begin{array}{ll|}
	x = \sqrt[3]{-4} & (1)\\
	x = \sqrt[3]{3} & (2)
\end{array}\right.$ (1) Сразу можно выделить $x_1 = -\sqrt[3]{4}$. Осталось найти еще 2 корня. Так как в системе координат они образуют правильный многоугольник, а в нашем случае треугольник, то можно однозначно найти 2 точки, соответствующие нашим корням. В тригонометрической форму они имеют вид: $\sqrt[3]4\big(\cos(\pm\dfrac{\pi}{3})+i\sin(\pm\dfrac{\pi}{3})\big)$. И соответственно в алгебраической: $\left[\begin{array}{l|}
x_2 = \dfrac{\sqrt[3]4}{2} - \sqrt[3]4\dfrac{\sqrt3}{2}i\\[7pt]
x_3 = \dfrac{\sqrt[3]4}{2} + \sqrt[3]4\dfrac{\sqrt3}{2}i
\end{array}\right.$ (2) Сразу выделим $x_4 = \sqrt[3]{3}$. Осталось найти еще 2. Аналогично можно найти 2 точки на единичной окружности(только в этом случае треугольник будет развернут в другую сторону, так как действительная часть положительна). Итого в тригонометрической форме корни имеют вид: $\sqrt[3]3\big(\cos(\pm \dfrac{2\pi}{3}) + i\sin(\pm \dfrac{2\pi}{3})\big)$. А в алгебраической форме соответственно:$\left[\begin{array}{l}
x_5 = -\dfrac{\sqrt[3]3}{2} - \sqrt[3]3\dfrac{\sqrt3}{2}i\\[7pt]
x_6 = -\dfrac{\sqrt[3]3}{2} + \sqrt[3]3\dfrac{\sqrt3}{2}i
\end{array}\right.$ \\
Таким образом, $f(x) = (x-x_1)(x-x_2)(x-x_3)(x-x_4)(x-x_5)(x-x_6)$ -- в кольце $\Co[x]$. Причем $x_2 и x_3, x_5,x_6$ -- сопряженные. Зная, что в кольце $\R[x]$ неприводимыми являются линейные многочлены и многочлены, корни которых сопряженные комплексные числа($D < 0$), то легко получим разложение $f(x)$ в кольце $\R[x]$: $f(x) = (x - x_1)(x-x_4)(x^2 - 2^{2/3} x + 2^{4/3})(x^2 + 3^{1/3}x + 3^{2/3})$  






\begin{center}
	\fbox{Задание 3.}
\end{center}

\noindent \textit{Выясните, является ли число $5+\sqrt{-5}$ простым элементом кольца $\Z[\sqrt{-5}]$.}\\
\noindent \textbf{Решение.} Число $5+\sqrt{-5}$ является простым $\Leftrightarrow$ оно представило в виде произведения. Имеем $5+\sqrt{-5} = xy =(a + b\sqrt{-5})(c + d\sqrt{-5})$. Рассмотрим модуль комплексного числа. Пусть $N(z)$ -- квадрат модуля комплексного числа(не путать с нормой). Имеем $N(5+\sqrt{-5})= N(xy) = N(x)N(y) \Leftrightarrow30 = (a^2 + 5b^2)(c + 5d^2) = 1 \cdot 30 = 2 \cdot 15 = 3 \cdot 10 = 5\cdot 6$. Заметим, что $N(x) = 2$ можно получить только при $a = \pm\sqrt2, b = 0$, но тогда $a, b$ -- не целые. Аналогично и с $N(x) = 3$. Остается один вариант: $\begin{cases}
N(x) = 5\\N(y) = 6
\end{cases}$. Рассмотрим $N(x) = 5 \Leftrightarrow a^2 + 5b^2 = 5$. Одна из переменных очевидно 0. Пусть $b = 0$, тогда $c$ и $d$ -- не целые. Следовательно, $a = 0$, тогда $b = \pm 1$. Положим $b = 1$, тогда $5+\sqrt{-5} = \sqrt{-5}(c + d\sqrt{-5})$. Легко можно подобрать такие целые $c$ и $d$, что равенство становится верным. Например $c = 1, d = -1$:
$5+\sqrt{-5} = \sqrt{-5}(-\sqrt{-5} + 1)$. Теперь покажем, что получили разложение на произведение необратимых. 
\begin{enumerate}
	\item Пусть $\sqrt{-5}$ обратим, тогда $\sqrt{-5}(a + b\sqrt{-5}) = 1 \Leftrightarrow a\sqrt{-5} - 5b = 1 \Rightarrow b = -\dfrac{1}{5}$, но $b \in \Z$.
	\item Пусть $-\sqrt{-5} + 1$ обратим, тогда $(-\sqrt{-5} + 1)(a+b\sqrt{-5}) = 1 \Rightarrow a + 5b = 1$, так как $a, b \in \Z$, то единственный возможный вариант $a = 1, b = 0$, но тогда обратное число равно 1, чего быть не может.
\end{enumerate}
Данное число не является простым.
\end{spacing}



\begin{center}
	\fbox{Задание 4.}
\end{center}

\noindent \textit{Пусть $R$ -- евклидово кольцо с нормой $N$. Докажите, что $N$ принимает бесконечное число значений.}\\
\noindent \textbf{Решение.} Пусть $N$ принимает конечное число значений. Следовательно, существует и максимальное значение Отсюда $\exists a \in R: N(a) = N_{max}$. Так как $R$ -- не поле, то существует необратимый $b$. Так как $N$ -- норма, то выполнено следующее свойство:
$$
N(x) = N(xy) \Leftrightarrow \exists y^{-1}
$$
Но для элемента $a$ не может быть равенства $N(a) = N(ab)$, так как $b$ необратим. Следовательно, по определению нормы: $N(ab) > N(a) \Leftrightarrow N(ab) > N_{max}$ -- противоречие. 
\end{document}