\documentclass[a4paper, 12pt]{article}
\usepackage[T1]{fontenc}
\usepackage[utf8]{inputenc}
\usepackage{graphicx}
\usepackage{xcolor}


\usepackage[colorlinks=true]{hyperref}
\usepackage{tabularx}

\usepackage{amsmath,amssymb,amsthm,textcomp}
\usepackage{enumerate}
\usepackage{multicol}
\usepackage{tikz}
\usepackage[english, russian]{babel}

\usepackage{cases}

\usepackage{geometry}
\geometry{total={210mm,297mm},
	left=15mm,right=15mm,%
	bindingoffset=0mm, top=20mm,bottom=20mm}

\usepackage{setspace}





\newcommand{\R}{\mathbb{R}} 
\newcommand{\Z}{\mathbb{Z}}
\newcommand{\x}{\times}
\newcommand{\N}{\mathbb{N}} 
\newcommand{\Q}{\mathbb{Q}} 
\newcommand{\Co}{\mathbb{C}}
\newcommand{\al}{\alpha}
\newcommand{\GCD}{\text{НОД}}


\title{
     Домашнее задание по алгебре №7.
 }
 \author{Михайлов Никита Маратович, ПМИ-167.
}
\date{}


\begin{document}
\maketitle
\begin{spacing}{1}


%%%%%%%%%%%%%%%%%%%%%%%%%%%%%%%%%%%%%%%%%%%%%%%%%%%%%%%%%%%%
%%%%%%%%%%%%%%%%%%%%%%%%%%%%%%%%%%%%%%%%%%%%%%%%%%%%%%%%%%%%
%%%%%%%%%%%%%%%%%%%%%%%%%%%%%%%%%%%%%%%%%%%%%%%%%%%%%%%%%%%%

\begin{center}
	\fbox{Задание 1.}
\end{center}

\noindent \textit{Пусть $\alpha$ -- комплексный корень многочлена $x^3-3x+1$. Представьте элемент
	$$
	\frac{\al^4 - \al^3+4\al+3}{\al^4+\al^3-2\al^2+1} \in \Q(\al)
	$$
	в виде $f(\al)$, где $f(x) \in \Q[x]$ и $deg\:f(x) \leq 2$.	
}\\
\noindent \textbf{Решение.} Так как $\al$ -- корень данного многочлена, то $A(\al) = \al^3-3\al+1 = 0$. Поэтому разделим числитель и знаменатель на $A(\al)$. Тогда исходная дробь будет равна дроби, числитель и знаменатель которой есть остатки от деления на $A(\al)$ соответственно. \\
Заметим, что 1) $\al^4 - \al^3+4\al+3 = (\al - 1)A(\al) + 3\al^2+4$; 2) $\al^4+\al^3-2\al^2+1 = (\al+1)A(\al) + \al^2+2\al$. Следовательно, исходная дробь ранва дроби $\dfrac{3\al^2+4}{\al^2+2\al}$. Пусть $g(\al)=\al^2+2\al$, тогда $f(\al) = (3\al^2+4) \cdot g^{-1}(\al)$. Найдем $g^{-1}(\al)$.\\
\textit{Примечание. } $\GCD(a,b) = 1 \Rightarrow \exists x,y: ax+by=1$, где $a, b, x, y$ -- многочлены. Но если $\al$ -- корень $a$, то $ax = 0$. Получим $by = 1$, откуда следует, что $y = b^{-1}$\\
Теперь заметим, что $\GCD(A(\al), g(\al)) = -1$ (дальше все равно будет видно, что действительно -1). Применим  расширенный алгоритм Евклида:
\begin{enumerate}
	\item $\al^3-3\al+1 = (\al^2+2\al)(\al-2) + (\al + 1) \Leftrightarrow (\al+1)=\al^3-3\al+1 - (\al^2+2\al)(\al-2) = A(\al)-(\al-2)g(\al)$
	\item $\al^2+2\al = (\al+1)(\al+1) - 1 \Leftrightarrow 1 = (\al+1)^2-(\al^2+2\al) = (A(\al)-(\al-2)g(\al))(\al+1) - g(\al)$. Применим тот факт, что $A(\al) = 0$, тогда $1 = (2-\al)g(\al)(\al + 1)- g(\al) = g(\al)((2-\al)(\al+1)-1) = g(\al)(-\al^2+\al+1) = 1$. 
\end{enumerate}
Подставим полученный многочлен вместо дроби: $(3\al^2+4)(-\al^2+\al+1) = -3\al^4+3\al^3-\al^2+4\al+4$. Возьмем остаток от делания на $A(\al)$. Получим:$-10\al^2+16\al+1$ -- искомый элемент.
%%%%%%%%%%%%%%%%%%%%%%%%%%%%%%%%%%%%%%%%%%%%%%%%%%%%%%%%%%%%
%%%%%%%%%%%%%%%%%%%%%%%%%%%%%%%%%%%%%%%%%%%%%%%%%%%%%%%%%%%%
%%%%%%%%%%%%%%%%%%%%%%%%%%%%%%%%%%%%%%%%%%%%%%%%%%%%%%%%%%%%




\begin{center}
	\fbox{Задание 2.}
\end{center}

\noindent \textit{Найдите минимальный многочлен для числа $\sqrt{3}-\sqrt{5}$ над $\Q$
}\\
\noindent \textbf{Решение.} Пусть $x = \sqrt{3}-\sqrt{5}$, тогда $x^2 = 8 - 2\sqrt{15} \Leftrightarrow 2\sqrt{15} = 8-x^2$. Составим \\$60 = (x^2-8)^2 \Leftrightarrow x^4-16x^2+4 = 0$. Покажем, что это минимальный многочлен, убедившись, что он неприводим над полем $\Q$. Решим уравнение $x^4-16x^2+4 = 0$. Обозначим $x^2 = t \geq 0$, тогда $t^2 - 16t + 4 = 0 \Rightarrow t_{1,2} = \dfrac{8\pm\sqrt{64-4}}{1}=8\pm 2\sqrt{15}\Rightarrow 
\left[\begin{array}{l}
	t_1 = 8 - 2\sqrt{15}\\
	t_2 = 8 + 2\sqrt{15}
\end{array}\right.$ Перейдем к $x$: $\left[\begin{array}{l}
x^2 = 8 - 2\sqrt{15}\\
x^2 = 8 + 2\sqrt{15}
\end{array}\right.\Rightarrow \left[\begin{array}{l}
	x_1 = \sqrt{3}-\sqrt{5}\\
	x_2 = \sqrt{5}-\sqrt{3}\\
	x_3 = \sqrt{3}+\sqrt{5}\\
	x_4 = -\sqrt{3}-\sqrt{5}
\end{array}\right.$ Корни получились иррациональными, а, следовательно, многочлен $x^4-16x^2+4$ неприводим и минимален над $\Q$.
%%%%%%%%%%%%%%%%%%%%%%%%%%%%%%%%%%%%%%%%%%%%%%%%%%%%%%%%%%%%
%%%%%%%%%%%%%%%%%%%%%%%%%%%%%%%%%%%%%%%%%%%%%%%%%%%%%%%%%%%%
%%%%%%%%%%%%%%%%%%%%%%%%%%%%%%%%%%%%%%%%%%%%%%%%%%%%%%%%%%%%


\begin{center}
	\fbox{Задание 3.}
\end{center}

\noindent \textit{Пусть $F$ -- подполе в $\Co$, полученное присоединением к $\Q$ всех комплексных корней многочлена $x^4+x^2+1$ (то есть $F$ -- наименьшее подполе в $\Co$, содержащее $\Q$ и все корни этого многочлена). Найдите степень расширения $[F:\Q]$.
}\\
\noindent \textbf{Решение.} Заметим, что $x^6 - 1 = (x^2 - 1)(x^4+x^2+1)$ -- разность кубов. Поэтому корнями данного многочлена будут все комплексные корни из 1 кроме $\pm 1$. Все корни можно получить возведением числа $\dfrac{1}{2} + \dfrac{\sqrt3}{2}i$ в натуральную степень (так как возведение в степень комплексного числа по модулю равного единицу -- это есть умножение его угла на степень). Самое маленькое подполе в $\Co$, содержащее корни данного многочлена совпадает с $\Q(\dfrac{1}{2} + \dfrac{\sqrt3}{2}i)$. Если рассматривать это поле как векторное пространство, то его размерность -- 2.
%%%%%%%%%%%%%%%%%%%%%%%%%%%%%%%%%%%%%%%%%%%%%%%%%%%%%%%%%%%%
%%%%%%%%%%%%%%%%%%%%%%%%%%%%%%%%%%%%%%%%%%%%%%%%%%%%%%%%%%%%
%%%%%%%%%%%%%%%%%%%%%%%%%%%%%%%%%%%%%%%%%%%%%%%%%%%%%%%%%%%%
\begin{center}
	\fbox{Задание 4.}
\end{center}

\noindent \textit{Пусть $F = \Co(x)$ -- поле рациональных дробей и $K = \Co(y)$, где $y = x + 1/x$. Найдите степень расширения $[F:K]$.
}\\
\noindent \textbf{Решение.} Заметим, что $x$ -- корень уравнения $x^2-xy+1 = 0$ над полем $\Co(y)$. Действительно, если подставить вместо $y$ число $x+1/x$, то получим $x^2-x(x+1/x) + 1 = x^2-x^2-1+1 = 0$ -- верно. Тогда если $x$ не является элементом $\Co(y)$, то степень расширения равна двум(так как через базис $\Co(y)$ нельзя будет выразить $x$). Пусть $x \in \Co(y)$, тогда $\exists P(y), Q(y)$ такие, что $x = \dfrac{P(y)}{Q(y)}$. Тогда рассмотрим предел левой и правой частей при $x \to \pm i$. Тогда $\lim\limits_{x \to \pm i}x+\dfrac{1}{x} =\lim\limits_{x \to \pm i}\dfrac{x^2+1}{x} = 0$. Тогда предел отношения $P$ к $Q$ будет стремиться либо к 0, либо к бесконечности, либо к отношению свободных членов. Левая часть не стремится ни к нулю, ни к бесконечности. Тогда правая часть тоже не должна к ним стремиться. Тогда очевидно, что если правая часть стремится к отношению свободных членов, то она стремится к такому же число и при замене $x \to -i$. Но левая часть меняет при этом знак. Получили противоречие. Следовательно, степень расширения равна 2.
%%%%%%%%%%%%%%%%%%%%%%%%%%%%%%%%%%%%%%%%%%%%%%%%%%%%%%%%%%%%
%%%%%%%%%%%%%%%%%%%%%%%%%%%%%%%%%%%%%%%%%%%%%%%%%%%%%%%%%%%%
%%%%%%%%%%%%%%%%%%%%%%%%%%%%%%%%%%%%%%%%%%%%%%%%%%%%%%%%%%%%


\end{spacing}
\end{document}