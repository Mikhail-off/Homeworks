\documentclass[a4paper, 10pt]{article}
\usepackage[T1]{fontenc}
\usepackage[utf8]{inputenc}
\usepackage{graphicx}
\usepackage{xcolor}


\usepackage[colorlinks=true]{hyperref}
\usepackage{tabularx}

\usepackage{amsmath,amssymb,amsthm,textcomp}
\usepackage{enumerate}
\usepackage{multicol}
\usepackage{tikz}
\usepackage[english, russian]{babel}

\usepackage{cases}

\usepackage{geometry}
\geometry{total={210mm,297mm},
	left=15mm,right=15mm,%
	bindingoffset=0mm, top=20mm,bottom=20mm}

\usepackage{setspace}





\newcommand{\R}{\mathbb{R}} 
\newcommand{\Z}{\mathbb{Z}}
\newcommand{\x}{\times}
\newcommand{\N}{\mathbb{N}} 
\newcommand{\Q}{\mathbb{Q}} 
\newcommand{\Co}{\mathbb{C}}
\newcommand{\F}{\mathbb{F}}
\newcommand{\al}{\alpha}
\newcommand{\GCD}{\text{НОД}}


\title{
	Домашнее задание по алгебре №9.
}
\author{Михайлов Никита Маратович, ПМИ-167.
}
\date{}


\begin{document}
\maketitle
\begin{spacing}{1}
%%%%%%%%%%%%%%%%%%%%%%%%%%%%%%%%%%%%%%%%%%%%%%%%%%%%%%%%%%%%
%%%%%%%%%%%%%%%%%%%%%%%%%%%%%%%%%%%%%%%%%%%%%%%%%%%%%%%%%%%%
%%%%%%%%%%%%%%%%%%%%%%%%%%%%%%%%%%%%%%%%%%%%%%%%%%%%%%%%%%%%
		
\begin{center}
	\fbox{Задание 1.}
\end{center}
		
\noindent \textit{Постройте явно поле $\F_8$ и составьте для него таблицы сложения и умножения.
}\\
\noindent \textbf{Решение.} Заметим, что $8=2^3$. Следовательно, нам подойдет поле $\Z_2[x]/(x^3+x^2+1)$, т.к. $x^3+x^2+1$ -- неприводим над $\Z_2$. Элементами поля будут многочлены вида $ax^2+bx+x, a,b,c\in\Z_2$. Построим таблицу сложения:
$$
\begin{array}{|c|c|c|c|c|c|c|c|c|}
\hline
+			& 0 & 1 & x & x + 1 & x^2 & x^2 + x & x^2 + 1 &  x^2 + x + 1\\
\hline
0			& 0 & 1 & x & x + 1 & x^2 & x^2 + x & x^2 + 1 &  x^2 + x + 1\\
\hline
1			& 1 & 0 & x + 1 & x & x^2 + 1 & x^2 + x + 1 & x^2 &  x^2 + x\\
\hline
x			& x & x + 1 & 0 & 1 & x^2 + x & x^2 & x^2 + x + 1 &  x^2 + 1\\
\hline
x + 1		& x + 1 & x & 1 & 0 & x^2 + x + 1 & x^2 + 1 & x^2 + x &  x^2\\
\hline
x^2			& x^2 & x^2 + 1 & x^2 + x & x^2 + x + 1 & 0 & x & 1 & x + 1\\
\hline
x^2 + x		& x^2 + x & x^2 + x + 1 & x^2 & x^2 + 1 & x & 0 & x + 1 & 1\\
\hline
x^2 + 1		& x^2 + 1 & x^2 & x^2 + x + 1 & x^2 + x & 1 & x + 1 & 0 & x\\
\hline
x^2 + x + 1	& x^2 + x + 1 & x^2 + x & x^2 + 1 & x^2 & x + 1 & 1 & x & 0\\
\hline
\end{array}
$$

А теперь умножения:
$$
\begin{array}{|c|c|c|c|c|c|c|c|c|}
\hline
\times		& 0 & 1 & x & x + 1 & x^2 & x^2 + x & x^2 + 1 &  x^2 + x + 1\\
\hline
0			& 0 & 0 & 0 & 0 & 0 & 0 & 0 &  0\\
\hline
1			& 0 & 1 & x & x + 1 & x^2 & x^2 + x & x^2 + 1 &  x^2 + x + 1\\
\hline
x			& 0 & x & x^2 & x^2 + x & x^2 + 1 & 1 & x^2 &  x + 1\\
\hline
x + 1		& 0 & x + 1 & x^2 + x & x^2 + 1 & x + 1 & x^2 + x + 1 & x &  x^2\\
\hline
x^2			& 0 & x^2 & x^2 + 1 & 1 & x^2 + x + 1 & x & x + 1 &  x^2 + x\\
\hline
x^2 + x		& 0 & x^2 + x & 1 & x^2+x+1 & x & x + 1 & x^2 &  x^2 + 1\\
\hline
x^2 + 1		& 0 & x^2 + 1 & x^2 + x + 1 & x & x + 1 & x^2 & x^2 + x & 1\\
\hline
x^2 + x + 1	& 0 & x^2 + x + 1 & x + 1 & x^2 & x^2 + x & x^2 + 1 & 1 & x\\
\hline
\end{array}
$$
%%%%%%%%%%%%%%%%%%%%%%%%%%%%%%%%%%%%%%%%%%%%%%%%%%%%%%%%%%%%
%%%%%%%%%%%%%%%%%%%%%%%%%%%%%%%%%%%%%%%%%%%%%%%%%%%%%%%%%%%%
%%%%%%%%%%%%%%%%%%%%%%%%%%%%%%%%%%%%%%%%%%%%%%%%%%%%%%%%%%%%
		
\begin{center}
	\fbox{Задание 2.}
\end{center}
		
\noindent \textit{Реализуем поле $\F_9$ в виде $\Z_3[x]/(x^2 + 1)$. Перечислите в этой реализации все элементы данного поля, являющиеся порождающими циклической группы $\F_9^\times$.
}\\
\noindent \textbf{Решение.} Решим в лоб. Построим таблицу умножения:
$$
\begin{array}{|c|c|c|c|c|c|c|c|c|c|}
\hline
\times	& 0 & 1 & 2 & x & x + 1 & x + 2 & 2x & 2x + 1 & 2x + 2\\
\hline
0		& 0 & 0 & 0 & 0 & 0 & 0 & 0 & 0 & 0\\
\hline
1		& 0 & 1 & 2 & x & x + 1 & x + 2 & 2x & 2x + 1 & 2x + 2\\
\hline
2		& 0 & 2 & 1 & 2x & 2x + 2 & 2x + 1 & x & x + 2 & x + 1\\
\hline
x		& 0 & x & 2x & 2 & x + 2 & 2x + 2 & 1 & x + 1 & 2x + 1\\
\hline
x + 1	& 0 & x + 1 & 2x + 2 & x + 2 & 2x & 1 & 2x + 1 & 2 & x\\
\hline
x + 2	& 0 & x + 2 & 2x + 1 & 2x + 2 & 1 & x & x + 1 & 2x & 2\\
\hline
2x		& 0 & 2x & x & 1 & 2x + 1 & 1 + x & 2 & 2x + 2 & x + 2\\
\hline
2x + 1	& 0 & 2x + 1 & x + 2 & x + 1 & 2 & 2x & 2x + 2 & x & 1\\
\hline
2x + 2	& 0 & 2x + 2 & x + 1 & 2x + 1 & x & 2 & x + 2 & 1 & 2x\\
\hline
\end{array}
$$
Выпишем элементы, порядок которых равен 8: $x + 1, x + 2, 2x + 1, 2x + 2$ -- эти элементы и будут порождающими.
%%%%%%%%%%%%%%%%%%%%%%%%%%%%%%%%%%%%%%%%%%%%%%%%%%%%%%%%%%%%
%%%%%%%%%%%%%%%%%%%%%%%%%%%%%%%%%%%%%%%%%%%%%%%%%%%%%%%%%%%%
%%%%%%%%%%%%%%%%%%%%%%%%%%%%%%%%%%%%%%%%%%%%%%%%%%%%%%%%%%%%
		
\begin{center}
	\fbox{Задание 3.}
\end{center}		
\noindent \textit{Проверьте, что многочлены $x^2 + 1$ и $y^2 - y - 1$ неприводимы над $\Z_3$, и установите явно изоморфизм между полями $\Z_3[x]/(x^2+1)$ и $\Z_3[y]/(y^2-y-1)$.
}\\
\noindent \textbf{Решение.} Так как поле конечно и достаточно маленькое. Давайте просто попробуем подобрать корни для $x^2 + 1 = 0$. Числа 0, 1, 2 -- не подходят.
Заметим, что $y^2 - x - 1 = y^2 + 2x + 2 = (y + 1)^2 + 1$ -- аналогичная ситуация. Корней нет. \\
Построим таблицу умножения и сделаем там, чтобы элементы одного порядке переходили друг в друга с выполнением линейности(коэфициенты перед $x$ и $y$ одинаковые, чтобы сохранить линейность):
$$
\begin{array}{|c|c|c|c|c|c|c|c|c|c|}
	\hline
	\times	& 0 & 1 & 2 & y & y + 1 & y + 2 & 2y & 2y + 1 & 2y + 2\\
	\hline
	0		& 0 & 0 & 0 & 0 & 0 & 0 & 0 & 0 & 0\\
	\hline
	1		& 0 & 1 & 2 & y & y + 1 & y + 2 & 2y & 2y + 1 & 2y + 2\\
	\hline
	2		& 0 & 2 & 1 & 2y & 2y + 2 & 2y + 1 & y & y + 2 & y + 1\\
	\hline
	y		& 0 & y & 2y & y + 1 & 2y + 1 & 1 & 2y+2 & 2 & y + 2\\
	\hline
	y + 1	& 0 & y + 1 & 2y + 2 & 2y + 1 & 2 & y & y + 2 & 2y & 1\\
	\hline
	y + 2	& 0 & y + 2 & 2y + 1 & 1 & y & 2y+2 & 2 & y + 1 & 2y\\
	\hline
	2y		& 0 & 2y & y & 2y+2 & y +2  & 2 & y+1 & 1 & 2y+1\\
	\hline
	2y + 1	& 0 & 2y + 1 & y + 2 & 2 & 2y & y+1 & 1 & 2y+2 & y\\
	\hline
	2y + 2	& 0 & 2y + 2 & y + 1 & y+2 & 1 & 2y & 2x + 1 & y & 2\\
	\hline
\end{array}
$$
\begin{enumerate}
	\item $\varphi(0) = 0$
	\item $\varphi(1) = 1$
	\item $\varphi(2) = 2$
	\item $\varphi(x) = y + 1$
	\item $\varphi(x + 1) = y + 2$
	\item $\varphi(x + 2) = y$
	\item $\varphi(2x) = 2y + 2$
	\item $\varphi(2x + 1) = 2y$
	\item $\varphi(2x + 2) = 2y + 1$
\end{enumerate}
%%%%%%%%%%%%%%%%%%%%%%%%%%%%%%%%%%%%%%%%%%%%%%%%%%%%%%%%%%%%
%%%%%%%%%%%%%%%%%%%%%%%%%%%%%%%%%%%%%%%%%%%%%%%%%%%%%%%%%%%%
%%%%%%%%%%%%%%%%%%%%%%%%%%%%%%%%%%%%%%%%%%%%%%%%%%%%%%%%%%%%
		
\begin{center}
	\fbox{Задание 4.}
\end{center}
		
\noindent \textit{Пусть $p$ -- простое число, $q = p^n$ и $\al\in\F_q$. Докажите, что если многочлен $x^p-x-\al \in \F_q[x]$ имеет корень, то он разлагается на линейные множители.
}\\
\noindent \textbf{Решение.} 
		
\end{spacing}
\end{document}