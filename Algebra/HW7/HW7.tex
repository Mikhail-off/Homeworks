\documentclass[a4paper, 12pt]{article}
\usepackage[T1]{fontenc}
\usepackage[utf8]{inputenc}
\usepackage{graphicx}
\usepackage{xcolor}


\usepackage[colorlinks=true]{hyperref}
\usepackage{tabularx}

\usepackage{amsmath,amssymb,amsthm,textcomp}
\usepackage{enumerate}
\usepackage{multicol}
\usepackage{tikz}
\usepackage[english, russian]{babel}

\usepackage{cases}

\usepackage{geometry}
\geometry{total={210mm,297mm},
	left=15mm,right=15mm,%
	bindingoffset=0mm, top=20mm,bottom=20mm}

\usepackage{setspace}





\newcommand{\R}{\mathbb{R}} 
\newcommand{\Z}{\mathbb{Z}}
\newcommand{\x}{\times}
\newcommand{\N}{\mathbb{N}} 
\newcommand{\Co}{\mathbb{C}}



\title{
     Домашнее задание по алгебре №7.
 }
 \author{Михайлов Никита Маратович, ПМИ-167.
}
\date{}


\begin{document}
\maketitle
\begin{spacing}{1}


%%%%%%%%%%%%%%%%%%%%%%%%%%%%%%%%%%%%%%%%%%%%%%%%%%%%%%%%%%%%
%%%%%%%%%%%%%%%%%%%%%%%%%%%%%%%%%%%%%%%%%%%%%%%%%%%%%%%%%%%%
%%%%%%%%%%%%%%%%%%%%%%%%%%%%%%%%%%%%%%%%%%%%%%%%%%%%%%%%%%%%

\begin{center}
	\fbox{Задание 1.}
\end{center}

\noindent \textit{Выразите симметрический многочлен
	$$
	f(x_1,x_2,x_3,x_4)=(x_1+x_2)(x_1+x_3)(x_1+x_4)(x_2+x_3)(x_2+x_4)(x_3+x_4)
	$$
	через элементарные симметрические многочлены.
}\\
\noindent \textbf{Решение.} Применим метод неопределенных коэффициентов. Заметим, что старший член равен $x_1^3x_2^2x_3^1x_4^0$. Переберем все возможные неубывающие наборы неотрицательных целых чисел, дающих в сумме 6, причем первое число не больше 3:
$$
(3,3,0,0)\qquad (3,2,1,0)\qquad(3,1,1,1)\qquad (2,2,1,1)\qquad(2,2,2,0)
$$
Пусть набору $x_1,x_2,x_3,x_4$ соответствует многочлен из произведения элементарных симметрических многочленов $\sigma_1^{x_1-x_2}\sigma_2^{x_2-x_3}\sigma_3^{x_3-x_4}\sigma_4^{x_4}$. Тогда $f$ представим в виде линейной комбинации многочленов, соответствующие найденным наборам. Составим представление из неопределенных коэффициентов(мы знаем, где будет старший член лежать, поэтому перед ним коэффициент равен 1):
\begin{gather*}
A\sigma_1^{3-3}\sigma_2^{3-0}\sigma_3^{0-0}\sigma_4^{0}+\sigma_1^{3-2}\sigma_2^{2-1}\sigma_3^{1-0}\sigma_4^{0}+B\sigma_1^{3-1}\sigma_2^{1-1}\sigma_3^{1-1}\sigma_4^{1}+C\sigma_1^{2-2}\sigma_2^{2-1}\sigma_3^{1-1}\sigma_4^{1}+D\sigma_1^{2-2}\sigma_2^{2-2}\sigma_3^{2-0}\sigma_4^{0}\\
=A\sigma_2^{3}+\sigma_1\sigma_2\sigma_3+B\sigma_1^2\sigma_4+C\sigma_2\sigma_4+D\sigma_3^2 = f(x_1,x_2,x_3,x_4)
\end{gather*}
Составим систему линейных уравнений, задавая конкретные значения для $x_i$:
$$
\begin{array}{c|c|c|c|c|c|c|c|c}
x_1 & x_2 & x_3 & x_4 & \sigma_1 & \sigma_2 & \sigma_3 & \sigma_4 & f\\
\hline
1 & 1 & 0 & 0 & 2 & 1 & 0 & 0 & 0\\
\hline
1 & 1 & 1 & 0 & 3 & 3 & 1 & 0 & 8\\
\hline
1 & 1 & 1 & 1 & 4 & 6 & 4 & 1 & 64\\
\hline
1 & 1 & -1 & -1 & 0 & -2 & 0 & 1 & 0
\end{array}
$$
На основе таблицы составим следующую систему:
$$
\begin{cases}
A = 0\\
27A+9+D=8\\
2 (108 A + 8B + 3C + 8D + 48)=64\\
-8 A - 2 C = 0
\end{cases} \Rightarrow \begin{cases}
A = 0\\
B=-1\\
C=0\\
D=-1
\end{cases}
$$
Таким образом искомое представление:
$$
f(x_1,x_2,x_3,x_4)=\sigma_1\sigma_2\sigma_3-\sigma_1^2\sigma_4-\sigma_3^2
$$
%%%%%%%%%%%%%%%%%%%%%%%%%%%%%%%%%%%%%%%%%%%%%%%%%%%%%%%%%%%%
%%%%%%%%%%%%%%%%%%%%%%%%%%%%%%%%%%%%%%%%%%%%%%%%%%%%%%%%%%%%
%%%%%%%%%%%%%%%%%%%%%%%%%%%%%%%%%%%%%%%%%%%%%%%%%%%%%%%%%%%%






\newpage
\begin{center}
	\fbox{Задание 2.}
\end{center}

\noindent \textit{Пусть $x_1, x_2, x_3$ -- все комплексные  корни многочлена $3x^3+2x^2-1$. Найдите значение этого выражения
$$
\frac{x_1x_2}{x_3}+\frac{x_1x_3}{x_2}+\frac{x_2x_3}{x_1}
$$}\\
\noindent \textbf{Решение.} По теореме Виета:\\
Если $\displaystyle x_{1},x_{2},x_{3}$  — корни кубического уравнения $\displaystyle p(x)=ax^{3}+bx^{2}+cx+d=0$ , то
$$
\begin{cases} x_1 + x_2 + x_3 = -\dfrac{b}{a}=-\dfrac{2}{3}\\[12pt]
x_1 x_2 + x_1 x_3 + x_2 x_3 = \dfrac{c}{a} =0\\[12pt]
x_1 x_2 x_3 = -\dfrac{d}{a} = \dfrac{1}{3}
\end{cases}
$$
Теперь представим нашу дробь через найденные выражения:
\begin{gather*}
\frac{x_1x_2}{x_3}+\frac{x_1x_3}{x_2}+\frac{x_2x_3}{x_1} = \frac{x_1^2x_2^2+x_1^2x_3^2+x_2^2x_3^2}{x_1x_2x_3} = \frac{(x_1x_2+x_1x_3+x_2x_3)^2-2x_1x_2x_3(x_1+x_2+x_3)}{x_1x_2x_3} =\\
= \frac{0+\frac{2}{3}\cdot\frac{2}{3}}{\frac{1}{3}} = \frac{4}{3}
\end{gather*}
%%%%%%%%%%%%%%%%%%%%%%%%%%%%%%%%%%%%%%%%%%%%%%%%%%%%%%%%%%%%
%%%%%%%%%%%%%%%%%%%%%%%%%%%%%%%%%%%%%%%%%%%%%%%%%%%%%%%%%%%%
%%%%%%%%%%%%%%%%%%%%%%%%%%%%%%%%%%%%%%%%%%%%%%%%%%%%%%%%%%%%





\begin{center}
	\fbox{Задание 3.}
\end{center}

\noindent \textit{Найдите многочлен 4-й степени, корнями которого является число 1 и кубы всех комплексных корней многочлена $x^3+x-1$}\\
\noindent \textbf{Решение.} По теореме Виета:
$$
\begin{cases} x_1 + x_2 + x_3 = 0\\[12pt]
x_1 x_2 + x_1 x_3 + x_2 x_3 = 1\\[12pt]
x_1 x_2 x_3 = 1
\end{cases}
$$
Выразим через найденные выражения многочлены:
\begin{enumerate}
	\item $x_1^3+x_2^3+x_3^3=(x_1+x_2)^3-3x_1x_2(x_1+x_2) + x_3^3 =(x_1+x_2+x_3)g(x_1,x_2,x_3) -3x_1x_2(x_1+x_2)=-3\dfrac{1}{x_3}(-x_3) = 3$
	\item $x_1^3x_2^3+x_1^3x_3^3+x_2^3x_3^3 = \sigma_2^3-3\sigma_1\sigma_2\sigma_3+3\sigma_3^2=4$
	\item $x_1^3x_2^3x_3^3=(x_1x_2x_3)^3=1$
\end{enumerate}
Тогда многочлен 4-й степени с корнями из условия выглядит следующим образом: $$
(x-1)(x^3-3x^2+4x-1) = x^4 - 4 x^3 + 7 x^2 - 5 x + 1
$$


%%%%%%%%%%%%%%%%%%%%%%%%%%%%%%%%%%%%%%%%%%%%%%%%%%%%%%%%%%%%
%%%%%%%%%%%%%%%%%%%%%%%%%%%%%%%%%%%%%%%%%%%%%%%%%%%%%%%%%%%%
%%%%%%%%%%%%%%%%%%%%%%%%%%%%%%%%%%%%%%%%%%%%%%%%%%%%%%%%%%%%







\begin{center}
	\fbox{Задание 4.}
\end{center}

\noindent \textit{Докажите, что не существует бесконечной последовательности одночленов от переменных $x_1, ..., x_n,$ в которой каждый последующий член строго меньше предыдущего в лексикографическом порядке.}\\
\noindent \textbf{Решение.} Применим метод математической индукции. База: $n = 1$. Чтобы последовательность была строго убывающей, надо уменьшить степень нашей единственной переменной. Любое сколь угодно большое натуральное число можно превратить в 1 за конечное число шагов.\\
Предположим верно для $n - 1$, покажем, что тогда верно и для $n$.\\
Рассмотрим последовательность. Возьмем самый старший член и посмотрим на степень $x_1$. Пусть его степень равна $k$. Тогда в последовательности остались одночлены со степенью $x_1$ равной $k$(коих конечно в сило того, что можно вынести $x_1^k$ за скобки и воспользоваться предположением индукции) или менее. Следовательно, степень $x_1$ уменьшится. На этом шаге мы отбросили конечное число одночленов. Покажем, что шагов будет конечно. Так как степень уменьшается, то мы дойдем до того момента, когда останутся только одночлены со степенью $x_1$ равной нулю за конечное число шагов. Но тогда остается $n - 1$ переменная, где по предположению индукции конечное число членов.




%%%%%%%%%%%%%%%%%%%%%%%%%%%%%%%%%%%%%%%%%%%%%%%%%%%%%%%%%%%%
%%%%%%%%%%%%%%%%%%%%%%%%%%%%%%%%%%%%%%%%%%%%%%%%%%%%%%%%%%%%
%%%%%%%%%%%%%%%%%%%%%%%%%%%%%%%%%%%%%%%%%%%%%%%%%%%%%%%%%%%%

\end{spacing}
\end{document}