\documentclass[a4paper, 12pt]{article}
\usepackage[T1]{fontenc}
\usepackage[utf8]{inputenc}
\usepackage{graphicx}
\usepackage{xcolor}


\usepackage[colorlinks=true]{hyperref}
\usepackage{tabularx}

\usepackage{amsmath,amssymb,amsthm,textcomp}
\usepackage{mathtools}
\usepackage{enumerate}
\usepackage{multicol}
\usepackage{tikz}
\usepackage[english, russian]{babel}

\usepackage{cases}
\usepackage{tasks}

\usepackage{geometry}
\geometry{total={210mm,297mm},
	left=15mm,right=15mm,%
	bindingoffset=0mm, top=20mm,bottom=20mm}

\usepackage{setspace}





\newcommand{\R}{\mathbb{R}} 
\newcommand{\Z}{\mathbb{Z}}
\newcommand{\x}{\times}
\newcommand{\N}{\mathbb{N}} 
\newcommand{\Q}{\mathbb{Q}} 
\newcommand{\Co}{\mathbb{C}}
\newcommand{\F}{\mathbb{F}}
\newcommand{\al}{\alpha}
\newcommand{\GCD}{\text{НОД}}
\newcommand{\D}{\displaystyle}


\title{
	Домашнее задание по теории вероятностей №2.
}
\author{Михайлов Никита Маратович, БПМИ-161.
}
\date{}


\begin{document}
\maketitle
\begin{spacing}{1}

\section{Задача №1.}
\textit{На воркшоп по взлому компьютерных сетей собрались $n$ программистов. После бурного семинара ни один из них не смог узнать свой ноутбук, и они разобрали их наугад. Далее, каждый из них с вероятностью $p$ независимо от других мог потерять ноутбук по дороге домой. Найдите вероятность того, что ни один программист не принес домой свой ноутбук.}
\subsection{Решение.} Заметим, что если программист взял свой ноутбук, то он должен его потерять, а если взял чужой, то все хорошо.\\
Пусть свой ноутбук взяло свой ноутбук ровно $k$ человек(их нужно еще выбрать), тогда у остальных $!(n-k)$ (кол-во беспорядков, как в прошлом д/з) способов распределить ноутбуки между собой так, чтобы никому не достался свой. Итого способов для конкретного $k$: $\binom{n}{k}\cdot !(n-k)$. Ну а всего способов $n!$. Вычислим вероятность для события "ровно $k$ человек взяло свой ноутбук $\binom{n}{k}\cdot \dfrac{!(n-k)}{n!}$. Но по нашему условию они их должны потерять, поэтому домножим на $p^k$. Итого для всех $k$ возьмем сумму по правилу сложения: 
$$
P = \sum\limits_{k=0}^n\binom{n}{k}\cdot \dfrac{!(n-k)}{n!}\cdot p^k
$$
\section{Задача №2.}
\textit{В легкоатлетической сборной выступают четыре спортсмена A, B, C, D. Спортсмен A принимает допинг перед соревнованиями с вероятностью $0,9$; спортсмен B — с вероятностью $0,5$; спортсмен C — с вероятностью $0,2$; а спортсмен D пытается победить честно и допинг не принимает (а потому всегда проигрывает). Антидопинговая	лаборатория выбрала случайного спортсмена и проверила его на допинг после двух	соревнований в течение года. Вероятность ошибки теста лаборатории при наличии допинга составляет $0,05$, а при его отсутствии результат теста всегда отрицательный. Получив два отрицательных результата, лаборатория решает протестировать	другого спортсмена, выбрав его случайно среди оставшихся. Известно, что в третий раз результат теста лаборатории был положительным. Каковы условные вероятности	того, что второй случайный спортсмен был, на самом деле, A, B, C или D, если все решения спортсменов (принимать или не принимать допинг перед соревнованиями) и результаты тестов независимы?}
\subsection{Решение.} Под <<условной вероятностью>> понимается
выполнение события $X_C =$ \{Вторым был спортсмен <<C>>\} при 
условии $M =$ \{Первый спортсмен прошел 2 допинг-пробы,
а второй завалил\}.
$$
P(X_C | M) = \dfrac{P(X_C \cap M)}{P(M)}
$$
Посчитаем $P(M)$. Для этого посчитаем вероятности пройти 2 пробы -- либо не было допинга, либо был, но ошиблась лаборатория и так 2 раза(а так же нужно выбрать спортсмена):
\begin{enumerate}
	\item $P(A_1) = \dfrac14\cdot(0.1 + 0.9 \cdot 0.05)^2$
	\item $P(B_1) = \dfrac14\cdot(0.5 + 0.5 \cdot 0.05)^2$
	\item $P(C_1) = \dfrac14\cdot(0.8 + 0.2 \cdot 0.05)^2$
	\item $P(D_1) = \dfrac14\cdot$
\end{enumerate}
Осталось посчитать вероятность завалить третью пробу -- принять допинг и отсутствие ошибки(и выбрать из оставшихся троих):
\begin{enumerate}
	\item $P(A_2) = \dfrac13\cdot0.9 \cdot 0.95$
	\item $P(B_2) = \dfrac13\cdot0.5 \cdot 0.95$
	\item $P(C_2) = \dfrac13\cdot0.2 \cdot 0.95$
	\item $P(D_2) = 0$
\end{enumerate}
Итого $P(M) = P(A_1)\cdot(P(B_2)+P(C_2)+P(D_2)) + P(B_1)\cdot(P(A_2)+P(C_2)+P(D_2)) + P(C_1)\cdot(P(B_2)+P(A_2)+P(D_2)) + P(D_1)\cdot(P(B_2)+P(C_2)+P(A_2)) = \frac{2.69462275}{12}$ -- спасибо python.\\
Теперь для каждого спортсмена посчитаем вероятность пересечения и разделим на $P(M)$:
\begin{enumerate}
	\item[A:] $P(X_A | M) = \dfrac{P(A_2)\cdot\bigg(P(B_1)+P(C_1) + P(D_1)\bigg)}{P(M)} = 0.6129336193721365$
	\item[B:] $P(X_B | M) = \dfrac{P(B_2)\cdot\bigg(P(A_1)+P(C_1) + P(D_1)\bigg)}{P(M)} =0.2956385545991549$
	\item[C:] $P(X_C | M) = \dfrac{P(C_2)\cdot\bigg(P(B_1)+P(A_1) + P(D_1)\bigg)}{P(M)} = 0.09142782602870847$
	\item[D:] $P(X_D | M) = 0$ 
	
\end{enumerate}
\end{spacing}
\end{document}
