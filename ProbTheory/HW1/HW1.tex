\documentclass[a4paper, 12pt]{article}
\usepackage[T1]{fontenc}
\usepackage[utf8]{inputenc}
\usepackage{graphicx}
\usepackage{xcolor}


\usepackage[colorlinks=true]{hyperref}
\usepackage{tabularx}

\usepackage{amsmath,amssymb,amsthm,textcomp}
\usepackage{mathtools}
\usepackage{enumerate}
\usepackage{multicol}
\usepackage{tikz}
\usepackage[english, russian]{babel}

\usepackage{cases}
\usepackage{tasks}

\usepackage{geometry}
\geometry{total={210mm,297mm},
	left=15mm,right=15mm,%
	bindingoffset=0mm, top=20mm,bottom=20mm}

\usepackage{setspace}





\newcommand{\R}{\mathbb{R}} 
\newcommand{\Z}{\mathbb{Z}}
\newcommand{\x}{\times}
\newcommand{\N}{\mathbb{N}} 
\newcommand{\Q}{\mathbb{Q}} 
\newcommand{\Co}{\mathbb{C}}
\newcommand{\F}{\mathbb{F}}
\newcommand{\al}{\alpha}
\newcommand{\GCD}{\text{НОД}}
\newcommand{\D}{\displaystyle}


\title{
	Домашнее задание по теории вероятностей №1.
}
\author{Михайлов Никита Маратович, БПМИ-161.
}
\date{}


\begin{document}
\maketitle
\begin{spacing}{1}
		
\section{Задача №1.} 
Известный польский математик Стефан Банах имел привычку носить в каждом из двух карманов пальто по коробку спичек. Всякий раз, когда ему хотелось закурить трубку, он выбирал наугад один из коробков и доставал из него спичку. Первоначально в каждом коробке было по $n$ спичек. Но когда-то наступает момент, когда выбранный наугад коробок оказывается пустым. Найдите вероятность того, что в этот момент времени во втором коробке осталось ровно $k = 0, . . . , n$ спичек.


\subsection{Решение.} Так как в условии написано "выбранный наугад коробок оказывается пустым", то будем считать, что после того, как математик выбрал последнюю спичку, ему нужно еще раз выбрать пустой коробок, и только после этого будем считать вероятность.\\
Раз во втором коробке осталось $k$ спичек, значит математик успел достать $2n - k$ спичек, а затем выбрал пустой коробок. Следовательно, вариантов закончить <<игру>> ровно $\binom{2n - k}{n - k}$, то есть выбрать $n - k$ мест для спичек из второй коробки. У одного такого варианта вероятность равна $\D \left(\frac{1}{2}\right)^{n - k}\cdot\left(\frac{1}{2}\right)^{n} = \frac{1}{2^{2n-k}}$. Так же в самом конце он должен выбрать пустой коробок с вероятностью $0.5$. Таким образом, нужное событие имеет вероятность $\D \frac{\binom{2n - k}{n - k}}{2^{2n-k+1}}$.\\
\hrule
\section{Задача №2.}
На шахматной доске размера $n \times n$ случайно размещают $n$ ладей. Найдите вероятности следующих событий:
\begin{enumerate}[(a)]
	\item $A = \{\text{ладьи не бьют друг друга}\}.$
	\item $B = \{\text{ладьи не бьют друг друга, и на главной диагонали нет никаких фигур}\}.$
	\item $C = \{\text{ладьи не бьют друг друга, и на главной диагонали находится ровно } t < n \text{ фигур}\}.$
\end{enumerate}
\subsection{Решение (a).}
У первой ладьи вариантов $n^2$, у второй уже $(n-1)^2$, у $i$-й. Итого нужных вариантов $n!^2$ -- для пронумерованных ладей, поэтому еще разделим на $n!$. Таким образом, $n!$ вариантов. Всего мы можем выбрать $\binom{n^2}{n}$. Следовательно, вероятность равна $\D \frac{n!}{\binom{n^2}{n}}$.
\subsection{Решение (b).} Пусть $M_i$ = \{кол-во расстановок, где на $i$-м месте на главной диагонали стоит ладья и ладьи не бьют друг друга\}, а так же обозначим $M$ = \{кол-во расстановок, где хотя бы одна ладья стоит на главной диагонали и ладьи не бью друг друга\} $$M =\bigcup\limits_{i = 1}^{n}M_i= \sum\limits_{k = 1}^{n}(-1)^{k+1}\sum\limits_{1\leq i_1\leq ... \leq i_k \leq n}\left(M_{i_1}\cap...\cap M_{i_k}\right) $$
Следовательно, $B$ есть $A - M$, где $A, B$ -- события из условия задачи, но в нашем контексте выступают как <<кол-во способов>>.\\
\begin{enumerate}
	\item Посчитаем $M_i$. Так как $i$-е место на диагонали занято, то остальным ладьям остается $(n - 1)^2$ мест. Таких способов $\D \frac{(n-1)!^2}{(n-1)!} = (n-1)!$ (см. пункт (a)).
	\item Посчитаем $M_{i_1}\cap...\cap M_{i_k}$. Так как $k$ мест на диагонали занято, то другим остается $(n-k)^2$ мест. Вариантов расставить остальных $(n-k)!$.
	\item Посчитаем M. $M=\sum\limits_{k = 1}^{n}(-1)^{k+1}\sum\limits_{1\leq i_1\leq ... \leq i_k \leq n}\left(M_{i_1}\cap...\cap M_{i_k}\right) = \sum\limits_{k=1}^n (-1)^{k+1}\cdot \binom{n}{k}\cdot(n-k)!$
	\item Посчитаем $B$. Составим $B = A - M = n! - \sum\limits_{k=1}^n (-1)^k\cdot \binom{n}{k}\cdot(n-k)! = n! + \sum\limits_{k=1}^n (-1)^k\cdot \binom{n}{k}\cdot(n-k)!$=\\=
	$n! + \sum\limits_{k=1}^n (-1)^k \dfrac{n!}{k!(n-k)!}(n-k)!=n! + \sum\limits_{k=1}^n (-1)^k \dfrac{n!}{k!} = \sum\limits_{k=0}^n(-1)^k\dfrac{n!}{k!} = !n$ (субфакториал)
	\item Посчитаем $\D P(B) = \frac{B}{\binom{n^2}{n}} = \frac{!n}{\binom{n^2}{n}}$
\end{enumerate}
\subsection{Решение (c).} Эта задача отличается от предыдущей тем, что некоторые $t$ мест на диагонали заняты, но заметим, что мы можем <<вырезать>> полностью свободные строки, столбцы и получим предыдущую задачу(не о вероятности, а о кол-ве способов) только для доски $(n-1) \times (n-t)$. Выбрать $t$ мест для главной диагонали можно $\binom{n}{t}$ способами. Кол-во способов расставить остальные фигурки равно $!(n-t)$. Итого $\D P(C) = \binom{n}{t}\frac{!(n-t)}{\binom{n^2}{n}}$.
\end{spacing}
\end{document}