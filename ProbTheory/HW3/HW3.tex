\documentclass[a4paper, 12pt]{article}
\usepackage[T1]{fontenc}
\usepackage[utf8]{inputenc}
\usepackage{graphicx}
\usepackage{xcolor}


\usepackage[colorlinks=true]{hyperref}
\usepackage{tabularx}

\usepackage{amsmath,amssymb,amsthm,textcomp}
\usepackage{mathtools}
\usepackage{enumerate}
\usepackage{multicol}
\usepackage{tikz}
\usepackage[english, russian]{babel}

\usepackage{cases}
\usepackage{tasks}

\usepackage{geometry}
\geometry{total={210mm,297mm},
	left=15mm,right=15mm,%
	bindingoffset=0mm, top=20mm,bottom=20mm}

\usepackage{setspace}





\newcommand{\R}{\mathbb{R}} 
\newcommand{\Z}{\mathbb{Z}}
\newcommand{\x}{\times}
\newcommand{\N}{\mathbb{N}} 
\newcommand{\Q}{\mathbb{Q}} 
\newcommand{\Co}{\mathbb{C}}
\newcommand{\F}{\mathbb{F}}
\newcommand{\al}{\alpha}
\newcommand{\GCD}{\text{НОД}}
\newcommand{\D}{\displaystyle}


\title{
	Домашнее задание по теории вероятностей №3.
}
\author{Михайлов Никита Маратович, БПМИ-161.
}
\date{}


\begin{document}
\maketitle
\begin{spacing}{1}

\section{Задача №1.}
Множество из $k$ шаров случайно раскладывают по $m$ ящикам. Случайная величина $\xi$ равна количеству пустых ящиков при таком случайном
размещении. Найдите $E\xi$ и $D\xi$, если (a) шары неразличимы, (b) шары различимы.
\subsection{Решение (a).} Найдем $P(\xi = i)$. Найдем число всех способов разложить $k$ шаров по $m$ ящиков. Введем переменные $x_i$ -- число шаров в $i$-ом ящике. Тогда искомое число способов равно числу способов решить в целых неотрицательных числах уравнение $x_1 + ... + x_m = k$. Как мы знаем, это задача Муавра, и число решений равно $\binom{k + m - 1}{k}$. Теперь найдем кол-во способов решить это уравнение так, что $i$ переменных равны нулю ($i$ ящиков пусты). Сначала выберем эти ящики: $\binom{m}{i}$. Теперь составим уравнение: $x_1 + ... + x_{m - i} = k$, но каждая переменная хотя бы 1, иначе у нас будет больше, чем $i$ нулевых переменных. Итого нужно найти число способов решить уравнение $y_1 + ... + y_{m - i} = k - (m - i)$. Снова задача Муавра, и ответ на нее есть $\D \binom{k - m + i + m - i - 1}{k - m + i} = \binom{k - 1}{k - m + i} = \binom{k - 1}{m - i - 1}$. Таким образом, $\D P(\xi = i) = \binom{m}{i}\cdot\dfrac{\binom{k - 1}{m - i - 1}}{\binom{k + m - 1}{k}}$. По свойству математического ожидания получим:
$$
E[\xi] = \sum\limits_{i \in \xi(\Omega)} i \cdot P(\xi = i) = \sum\limits_{i = 0}^{m} i \cdot \binom{m}{i}\cdot\dfrac{\binom{k - 1}{m - i - 1}}{\binom{k + m - 1}{k}}
$$
Найдем дисперсию:
$$
D\xi = E[\xi^2] - E^2[\xi] = \sum\limits_{i \in \xi(\Omega)} i^2 \cdot P(\xi = i) - \Big(\sum\limits_{i \in \xi(\Omega)} i \cdot P(\xi = i)\Big)^2 = 
$$
$$= \sum\limits_{i = 0}^{m} i^2 \cdot \binom{m}{i}\cdot\dfrac{\binom{k - 1}{m - i - 1}}{\binom{k + m - 1}{k}} - \Big(\sum\limits_{i = 0}^{m} i \cdot \binom{m}{i}\cdot\dfrac{\binom{k - 1}{m - i - 1}}{\binom{k + m - 1}{k}}\Big)^2
$$
\subsection{Решение (b).} Пусть мы как-то разложили шары, тогда перемешав шары, сохранив кол-во шаров в каждом из ящиков, мы можем получить $k!$ вариантов, но внутри каждого ящика порядок не важен. Пусть $(x_1, ...,x_m)$ -- некоторый способ разложения шаров(неразличимых). Из него получим $\dfrac{k!}{x_1!\cdot...\cdot x_m!}$ решений(для различимых). Пусть $X = \{x = (x_1,...,x_m)\;|\;x_1+...+x_m=k\}$ -- множество решений для неразличимых шаров. Итак, посчитаем $P(\xi=i)$. Для этого вычислим количество всех исходов: $\D \sum\limits_{x \in X}\dfrac{k!}{x_1!\cdot...\cdot x_m!}$. Теперь выберем $i$ пустых ящиков $\binom{m}{i}$ способами. Как и в аналогичной задаче, в остальных ящиках как минимум 1 шар есть. Составим задачу Муавра и пусть $Y = \{y=(y_1,...,y_{m-i})|y_1+...+y_{m-1} = k - m + i\}$ -- все ее решения для неразличимых шаров. Таким образом, $\binom{m}{i}\sum\limits_{y\in Y}\dfrac{k!}{y_1!\cdot...\cdot y_{m - i}!}$. Итого $\D P(\xi=i) = \dfrac{\binom{m}{i}\sum\limits_{y\in Y}\dfrac{k!}{y_1!\cdot...\cdot y_{m - i}!}}{\sum\limits_{x \in X}\dfrac{k!}{x_1!\cdot...\cdot x_m!}}$, а математическое ожидание величины $\xi$ соответственно:
$$
E[\xi] = \sum\limits_{i = 0}^{m} i\cdot \dfrac{\binom{m}{i}\sum\limits_{y\in Y}\dfrac{k!}{y_1!\cdot...\cdot y_{m - i}!}}{\sum\limits_{x \in X}\dfrac{k!}{x_1!\cdot...\cdot x_m!}} = \dfrac{\sum\limits_{i = 0}^{m} i\cdot \binom{m}{i}\sum\limits_{y\in Y}\dfrac{k!}{y_1!\cdot...\cdot y_{m - i}!}}{\sum\limits_{x \in X}\dfrac{k!}{x_1!\cdot...\cdot x_m!}} 
$$
Посчитаем дисперсию:
$$
D\xi = E[\xi^2] - E^2[\xi] = \sum\limits_{i \in \xi(\Omega)} i^2 \cdot P(\xi = i) - \Big(\sum\limits_{i \in \xi(\Omega)} i \cdot P(\xi = i)\Big)^2 = 
$$
$$
=\dfrac{\sum\limits_{i = 0}^{m} i^2\cdot \binom{m}{i}\sum\limits_{y\in Y}\dfrac{k!}{y_1!\cdot...\cdot y_{m - i}!}}{\sum\limits_{x \in X}\dfrac{k!}{x_1!\cdot...\cdot x_m!}} - \Bigg(\dfrac{\sum\limits_{i = 0}^{m} i\cdot \binom{m}{i}\sum\limits_{y\in Y}\dfrac{k!}{y_1!\cdot...\cdot y_{m - i}!}}{\sum\limits_{x \in X}\dfrac{k!}{x_1!\cdot...\cdot x_m!}}\Bigg)^2
$$
\section{Задача №2.}
В ящик положили $n$ различимых шаров, среди которых есть $k$ белых,
а остальные — черные. Шары вынимаются случайно и последовательно
без возвращения. Пусть последний белый шар вынимается на шаге $\xi$.
Вычислите а) $E\xi$, б) $D\xi$.
\subsection{Решение а)}
Так как в условии написано, что шары различимы, то будем считать, что они все пронумерованы и имеют цвет. Найдем вероятность $P(\xi = i)$. Понятно, что всех вариантов у нас $n!$. Раз последний белый шар вынут на $i$-ом шаге, то нужно выбрать из первых $i - 1$ мест $k - 1$ мест для белых шаров(ведь на $i$-м месте стоит один из $k$ белых шаров), а остальные $n - k$ мест распределить среди черных: $\D k\cdot A_{i-1}^{k-1} (n-k)!$. Таким образом $\D P(\xi = i) = \dfrac{k\cdot A_{i-1}^{k-1}(n-k)!}{n!}= \dfrac{k\cdot A_{i-1}^{k-1}}{A_n^k} = \dfrac{\binom{i-1}{k-1}}{\binom{n}{k}}$ , а математическое ожидание соответственно
$$
E[\xi] = \sum\limits_{i = k}^n i \cdot \dfrac{\binom{i-1}{k-1}}{\binom{n}{k}} = \dfrac{\sum\limits_{i = k}^n i \cdot \binom{i-1}{k-1}}{\binom{n}{k}}
$$
\subsection{Решение б)} По формуле связи дисперсии с математическим ожиданием получим:
$$
D\xi = E[\xi^2] - E^2[\xi] = \dfrac{\sum\limits_{i = k}^n i^2 \cdot \binom{i-1}{k-1}}{\binom{n}{k}} - \dfrac{(\sum\limits_{i = k}^n i \cdot \binom{i-1}{k-1})^2}{\binom{n}{k}^2} 
$$
\end{spacing}
\end{document}
